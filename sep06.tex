\subsection{06 сентября. д.р. Енукол, пер. Енукол}
\textit{Метеоусловия: весь день ясно, солнечно}

\begin{figure}[h!]
	\centering
	\includegraphics[angle=90, width=0.7\linewidth]{pics/maps/06}
	\label{fig:06}
\end{figure}

Подъём в 06:00. Завтракаем, распределяем еду и снарягу и в 07:50 выдвигаемся в путь.

\begin{figure}[h!]
	\centering
	\includegoogle[width=0.6\linewidth]{https://drive.google.com/file/d/1J9RD_3ZCU23Wi7bTvLzz8iI8Uye6OyzP/view?usp=drive_link}
	\caption{Вид с м.н. на д.р. Уллухурзук и а. Хурзук}
	\label{fig:camp05}
\end{figure}

За первые 30 мин ЧХВ выходим по дороге из зоны леса. Нас обгоняют вчерашние мотоциклисты и уезжают в сторону перевала. Немного выше зоны леса (2150 м) дорога заворачивает налево пхд к кошу, мы же поднимается по руслу пересыхающего ручья~--- правого притока р. Енукол. Подъём технически несложный, но отсутствие дорог и человеческих (коровьи -- в количестве) троп сказывается на скорости движения. Впрочем, в 11:05, к общей радости, мы выходим на дорогу у оборудованного родника (N 43.44810° E 42.21287°).

Дальнейшее движение не представляет сложности, ибо основной набор высоты позади. Идём по дороге, в 12:25 выходим на седловину пер. Енукол. Отгоняем Йошту от собаки пастуха, что пасёт здесь лошадей, любуемся потрясающим видом на Эльбрус.

\begin{figure}[h!]
	\centering
	\includegoogle[width=0.7\linewidth]{https://drive.google.com/file/d/1u5eQt-6JgZoyDq0462ldMXzdG3U2w2Qd/view?usp=drive_link}
	\caption{Группа на пер. Енукол. Вид на запад, в сторону пер. Быкылы}
	\label{fig:enukol}
\end{figure}

Выходим дальше в 12:45, и спустя 1 км пути, в 13:00, встаём на обед у левого притока р. Эльмезтебе, временами отгоняя покушающихся на нашу площадку лошадей. Координаты места обеда: N 43.43883° E 42.23184°

\begin{figure}[h!]
	\centering
	\includegoogle[width=0.7\linewidth]{https://drive.google.com/file/d/1MpMHBACfyQ9NZ3nVVavx38AQjCu7LgTy/view?usp=drive_link}
	\caption{Место обеда}
	\label{fig:06_dinner}
\end{figure}

Выходим в 14:30, движемся дальше по тропе на восток. Дорога отличная, открываются обзоры то на д.р. Эльмезтебе, то на д.р. Уллухурзук (в зависимости от того, по какой стороне хребта идём). 

\begin{figure}[h!]
	\centering
	\includegoogle[width=0.7\linewidth]{https://drive.google.com/file/d/1qcjR0S9gauWMbnehAhiaB5mdV23Z3s9E/view?usp=drive_link}
	\caption{Безымянная седловина у правого притока р. Уллухурзук}
	\label{fig:saddle}
\end{figure}


В 16:20 проходим запланированное место ночёвки, но не обнаруживаем там воды и решаем дойти до обозначенной на карте стоянки у правого притока р. Уллухурзук. По пути проходим живописную безымянную седловинку со скальными <<воротами>>.

\begin{figure}[h!]
	\centering
	\includegoogle[width=0.7\linewidth]{https://drive.google.com/file/d/1gEzYoahWEPCj-cKZEtuZC0grhR3LhpUd/view?usp=drive_link}
	\caption{Скальные <<ворота>> перед следующей безымянной седловиной}
	\label{fig:gate}
\end{figure}
 
Сразу за безымянным перевалом тропа поворачивает налево пхд и спускается на 50 вертикальных метров к месту ночёвки.

В 17:30 встаём у притока. Место оборудовано, выложены каменные ветрозащитные стенки, места хватит на 6--7 палаток. Любуемся золотым закатом на суровых ледопадах западного склона Эльбруса. Координаты м.н.: N 43.44205° E 42.28980°.

\begin{figure}[h!]
	\centering
	\includegoogle[width=0.4\linewidth]{https://drive.google.com/file/d/1JAVG5Qrq1IfGuimlA4p2fcPjJg05-N0-/view?usp=drive_link}
	\caption{Закат на м.н.}
	\label{fig:camp06}
\end{figure}

\clearpage