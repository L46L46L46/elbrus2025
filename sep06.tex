\subsection{06 сентября. д.р. Енукол, пер. Енукол}
\textit{Метеоусловия: весь день ясно, солнечно}

\begin{figure}[h!]
	\centering
	\includegraphics[angle=90, width=0.7\linewidth]{pics/maps/06}
	\label{fig:06}
\end{figure}

Подъём в 06:00. Завтракаем, распределяес еду и снарягу и в 07:50 выдвигаемся в путь.

\begin{figure}[h!]
	\centering
	\includegoogle[width=0.7\linewidth]{https://drive.google.com/file/d/1J9RD_3ZCU23Wi7bTvLzz8iI8Uye6OyzP/view?usp=drive_link}
	\caption{Вид с м.н. на д.р. Уллухурзук и а. Хурзук}
	\label{fig:camp05}
\end{figure}

За первые 30 мин ЧХВ выходим по дороге из зоны леса. Нас обгоняют вчерашние мотоциклисты и уезжают в сторону перевала. Немного выше зоны леса (2150 м) дорога заворачивает налево пхд к кошу, мы же поднимается по руслу пересыхающего ручья~--- правого притока р. Енукол. Подъём технически несложный, но отсутствие дорог и человеческих (коровьи -- в количестве) сказывается на скорости движения. Впрочем, в 11:05, к общей радости, мы выходим на дорогу у оборудованного родника (N 43.44810° E 42.21287°).

Дальнейшее движение не представляет сложности, ибо основной набор высоты позади. Идём по дороге, в 12:25 выходим на седловину пер. Енукол. Отгоняем Йошту от собаки пастуха, что пасёт здесь лошадей, любуемся потрясающим видом на Эльбрус.

\begin{figure}[h!]
	\centering
	\includegoogle[width=0.7\linewidth]{https://drive.google.com/file/d/1u5eQt-6JgZoyDq0462ldMXzdG3U2w2Qd/view?usp=drive_link}
	\caption{Группа на пер. Енукол. Вид на запад, в сторону пер. Быкылы}
	\label{fig:enukol}
\end{figure}





\clearpage