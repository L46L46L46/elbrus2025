\subsection{07 сентября. пер. Быкылы, пер. Чемарт}
\textit{Метеоусловия: утром ясно, днём переменная облачность, сильный дождь, вечером туман, переменная облачность, дождь}

\begin{figure}[h!]
	\centering
	\includegraphics[angle=90, width=0.8\linewidth]{pics/maps/07}
	\label{fig:07}
\end{figure}

\textit{Чемарт~--- карач.-балк. <<Чомарт>>~--- <<Щедрый>>.}


Подъём в 04:30, выход в 06:20. Продолжаем двигаться по тропе, траверсирующей правый пхд склон хребта Енукол. Утро чудесное, тропа пологая и прекрасная. Без проблем в 07:05 поднимаемся на седловину пер. Быкылы. Часть группы сочиняет перепевку песни <<Маленький рыцарь>>, сфоткаться на перевале забываем, однако есть фото с седловины в направлении дальнейшего движения:

\begin{figure}[h!]
	\centering
	\includegoogle[width=0.7\linewidth]{https://drive.google.com/file/d/143dSenn0k_XXosOCQXjsUuSXmiAiG8s7/view?usp=drive_link}
	\caption{Вид с пер. Быкылы на тропу Сют-Джол (на восток)}
	\label{fig:bykyly}
\end{figure}


В 07:25 выдвигаемся дальше по тропе. Теперь мы снова движемся по левому пхд склону хребта.

\begin{figure}[h!]
	\centering
	\includegoogle[width=0.7\linewidth]{https://drive.google.com/file/d/1V0cLi_dpn1h4Ez_8qMutj0Nw1QUKj-Nh/view?usp=drive_link}
	\caption{Подъём на пер. Чемарт}
	\label{fig:toward_chemart}
\end{figure}

В 08:45 поднимаемся на седловину пер. Чемарт. Фоткаемся и идём на спуск.

\begin{figure}[h!]
	\centering
	\includegoogle[width=0.8\linewidth]{https://drive.google.com/file/d/1vNLrlR8F2a7hRPu0-HLUm1inNfItQyVZ/view?usp=drive_link}
	\caption{Группа на пер. Чемарт, вид на запад}
	\label{fig:chemart_west}
\end{figure}


\begin{figure}[h!]
	\centering
	\includegoogle[width=0.7\linewidth]{https://drive.google.com/file/d/1trbzf2kwsnUP6fyRtxCgTHbHO5kqhSCc/view?usp=drive_link}
	\caption{Группа на пер. Чемарт, вид на восток}
	\label{fig:chemart_east}
\end{figure}


В этом месте читаемость тропы несколько снижается. Лучше всего забирать немного влево пхд, на небольшое выполаживание, которое на 80 горизонтальных метров ниже седловины. С выполаживания нужно повернуть круто направо, и там сразу снова начинает читаться тропа. Если же сразу с седловины забирать вправо (а это азимутальное направление), есть риск либо оказаться на крутом травянистом склоне, либо на моренном выносе со склона хребта (см. рис.~\ref{fig:chemart_scheme}). И то, и другое технически несложно, но преодоление\textsuperscript{TM} на ровном месте устраивать не нужно совершенно.


\begin{figure}[h!]
	\centering
	\includegoogle[width=0.9\linewidth]{https://drive.google.com/file/d/1pRS_BewomoutY5mU1S-LrSUAapUPNVoo/view?usp=drive_link}
	\caption{Путь спуска с пер. Чемарт. Если забрать вправо пхд, пожно нарваться на моренный вынос (на фото слева)}
	\label{fig:chemart_scheme}
\end{figure}

В 10:00 проходим запланированное м.н. (N 43.43122° E 42.36083°), на котором собирались, в силу обгона плана-графика, устроить обед, однако вода находится слишком далеко вниз от тропы для небольшой остановки (не менее 50 горизонтальных метров спуска), поэтому решаем дойти до следующей воды, которая находится в последнем перед д.р. Чемарткол распадке.

\begin{figure}[h!]
	\centering
	\includegoogle[width=0.9\linewidth]{https://drive.google.com/file/d/1O3t5Wcy7Pb_tlb69YcZRK1g1DGIYQs3R/view?usp=drive_link}
	\caption{Движение группы к месту обеда и далее к м.н.}
	\label{fig:sep07_route}
\end{figure}										

																
Приходим туда в 10:50 и устраиваем обед (N 43.42900° E 42.38536°). Почти сразу начинает накрапывать дождь, поэтому кушаем в темпе вальса (тем не менее, с горячим), и в 11:35 выходим дальше. 

Поскольку под ногами всё еще хорошая тропа, туман и сильный дождь не сильно сказываются на безопасности, хоть и расстраивают. Зато виды на д.р.~Чемарткол открываются завораживающие:

\begin{figure}[h!]
	\centering
	\includegoogle[width=0.9\linewidth]{https://drive.google.com/file/d/1kOj6LTTSpSDFAqlJkGaIDICWU_Qb_Uyq/view?usp=drive_link}
	\caption{Панорама д.р. Чемарткол после пер. Чемарт}
	\label{fig:chemart_panorama}
\end{figure}										

После огибания очередного отрожка хребта тропа становится менее читаемой и тонет в траве. Нам предстоит спуститься в д.р. Чемарткол, на 500 горизонтальных метров выше впадения в него правого притока, стекающего с пер. Бурунташ. Этот отрезок пути, наверное, был самым неприятным за весь поход, ибо нужно было спускаться по скользкой мокрой траве и земле в дождь, а руководитель несколько раз неуклюже катался попой по склону \smiley.

\begin{figure}[h!]
	\centering
	\includegoogle[width=0.8\linewidth]{https://drive.google.com/file/d/1GqBweevi9DQUAGfnRyVREgkf2BynKf4c/view?usp=drive_link}
	\caption{Спуск в д.р. Чемарткол}
	\label{fig:downhill}
\end{figure}



Так или иначе, в 12:30 спускаемся на дно долины и почти сразу встаём на приятных травяных площадках. Через пару часов дождь ненадолго прекращается, и мы даже успеваем просушить на ветру часть вещей. Вечером на нас наползает облако и снова начинается сильный дождь, который заканчивается аккурат к нашему подъёму в 4:30 утра.

Координаты м.н.: N 43.42883° E 42.39378°.

ЧХВ: 3:46, ОХВ: 06:10.
											
\clearpage