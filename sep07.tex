\subsection{07 сентября. пер. Быкылы, пер. Чемарт}
\textit{Метеоусловия: утром ясно, днём переменная облачность, сильный дождь, вечером туман, переменная облачность, дождь}

\begin{figure}[h!]
	\centering
	\includegraphics[angle=90, width=0.8\linewidth]{pics/maps/07}
	\label{fig:07}
\end{figure}

Подъём в 04:30, выход в 06:20. Продолжаем двигаться по тропе, траверсирующей правый пхд склон хребта Енукол. Утро чудесное, тропа пологая и прекрасная. Без проблем в 07:05 поднимаемся на седловину пер. Быкылы. Часть группы сочиняет перепевку песни <<Маленький рыцарь>>, сфоткаться на перевале забываем. В 07:25 выдвигаемся дальше по тропе.

В 08:45 поднимаемся на седловину пер. Чемарт. Фоткаемся и идём на спуск.



Теперь мы снова движемся по левому пхд склону хребтаю В этом месте читаемость тропы несколько снижается. Лучше всего забирать немного влево пхд, на небольшое выполаживание, которое на 80 горизонтальных метров ниже седловины. С выполаживания нужно повернуть круто направо, и там сразу снова начинает читаться тропа. Если же сразу с седловины забирать вправо (а это азимутальное направление), есть риск либо оказаться на крутом травянистом склоне, либо на моренном выносе со склона хребта. И то, и другое технически несложно, но ПрЕоДоЛеНиЕ на ровном месте устраивать не нужно совершенно.

																В \alert{СКОЛЬКО} проходим запланированное м.н., ИЛИ НЕТ на котором планировали, в силу обгона плана-графика, устроить обед, однако вода находится слишком далеко вниз от тропы для небольшой остановки (не менее 50 горизонтальных метров спуска), поэтому решаем дойти до следующей воды \alert{ГДЕ}. Приходим туда в 10:50 и устраиваем обед. Почти сразу начинает накрапывать дождь, поэтому кушаем в темпе вальса (тем не менее, с горячим), и в 11:35 выходим дальше. Поскольку под ногами всё еще хорошая тропа, сильный дождь и туман не сильно сказываются на безопасности, хоть и расстраивают. Зато виды на д.р.~\alert{КАКУЮ} открываются завораживающие:
																
																ФОТО ПАНОРАМЫ ДОЖДЬ
																

После огибания очередного отрожка хребта тропа становится менее читаемой и тонет в траве. Нам предстоит спуститься в д.р. Чемарткол, на 500~м выше впадения в него правого притока, стекающего с пер. Бурунташ. Этот отрезок пути, наверное, был самым неприятным за весь поход, ибо нужно было спускаться по скользкой мокрой траве и земле в дождь, а руководитель несколько раз неуклюже катался попой по склону \smiley. Так или иначе, в 12:30 спускаемся на дно долины и почти сразу встаём на приятных травяных площадках. Через пару часов дождь ненадолго прекращается, и мы даже успеваем просушить на ветру часть вещей. Вечером на нас наползает облако и снова начинается сильный дождь, который заканчивается аккурат к нашему подъёму в 4:30 утра.

Координаты м.н.: 

ЧХВ: , ОХВ: 06:10.
											
\clearpage