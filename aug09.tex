\subsection{9 августа. Подход под пер. Иттиш}
\textit{Метеоусловия: утром, днём, вечером ясно, тепло.}

Накануне участники, сошедшие с маршрута, забрали с собой, не считая личного снаряжения, 3 пары кошек, 3 ледоруба, тент и чуть меньше половины продуктов, рассчитанных на вторую часть похода. Вес рюкзаков у мужской части оставшихся участников в начале второй части похода составляет 25--27 кг.

\begin{figure}[h!]
	\centering
	\includegraphics[angle=0, width=0.4\linewidth]{pics/maps/9.png}
	\label{fig:mini_9}
\end{figure}

Подъем дежурных в 06:00, выходим с м.н. 5 на слиянии рек Джууку и Ашу-Кашкасу в 09:30. Проходим вдоль р. Джууку по ее орогр. правому берегу  по дороге и начинаем подъем вдоль ручья Иттиши по его орогр. левому берегу. Подъем идет по тропе по ельнику, угол наклона составляет 15-20\degree. В ходе подъема есть один крутой участок около 30м по перепаду высот, где тропа становится круче. Встаем на обед вблизи ручья в 13:00. Координаты места обеда N 42.05346\degree~ E 77.95250\degree.

\begin{figure}[h!]
	\centering
	\includegraphics[width=0.7\linewidth]{pics/09/IMG_3328}
	\caption{Тропа на подъеме на пер. Иттиш (вид сверху вниз)}
	\label{fig:IMG_3328}
\end{figure}

\begin{figure}[h!]
	\centering
	\includegraphics[width=0.7\linewidth]{pics/09/IMG_3346}
	\caption{Брод ручья и подъем к м.н. 6 по тр.-ос. склону}
	\label{fig:IMG_3346}
\end{figure}


После обеда начинаем движение в 14:50. Преодолеваем остаток подъема по лесу и выходим к средней осыпи. По ней поднимаемся еще на $\sim$100м, переходим ручей и, траверсируя с небольшим набором высоты осыпь и тр.-ос. склон, выходим к небольшому озеру. Встаем на ночевку вблизи озера в 17:30 на плановом м.н. 6. Координаты м.н. N 42.04430\degree~ E 77.95787\degree.


\begin{figure}[h!]
	\centering
	\includegraphics[width=0.7\linewidth]{pics/09/camp_09} % IMG_3357.JPG
	\caption{Место ночёвки 09--10.08}
	\label{fig:camp_09}
\end{figure}

\clearpage