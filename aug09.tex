\subsection{9 августа. Подход под пер. Иттиш}
\textit{Метеоусловия: утром, днём, вечером ясно, тепло.}

Накануне участники, сошедшие с маршрута, забрали с собой, не считая личного снаряжения, 3 пары кошек, 3 ледоруба, тент и чуть меньше половины продуктов, рассчитанных на вторую часть похода. Вес рюкзаков у мужской части оставшихся участников в начале второй части похода составляет 25-27 кг.

\begin{figure}[h!]
	\centering
	\includegraphics[angle=0, width=0.7\linewidth]{pics/maps/9}
	\alert{карта}
	\label{fig:mini_9}
\end{figure}

Выходим с м.н. 5 на слиянии рек Джууку и Ашу-Кашкасу \alert{около 09:00}. Проходим вдоль р. Джууку по дороге по орогр. правому берегу и начинаем подъем вдоль ручья Иттиши по его орогр. левому берегу. Подъем идет по тропе по ельнику, угол наклона составляет 15-20\degree. В ходе подъема есть один крутой участок около 40м по перепаду высот, где тропа становится круче. Встаем на обед вблизи ручья \alert{в 12:30}.

\begin{figure}[h!]
	\centering
	\includegraphics[width=0.7\linewidth]{pics/09/IMG_3328}
	\caption{Тропа на подъеме на пер. Иттиш (вид сверху вниз)}
	\label{fig:IMG_3328}
\end{figure}

\begin{figure}[h!]
	\centering
	\includegraphics[width=0.7\linewidth]{pics/09/IMG_3346}
	\caption{Брод ручья и подъем к м.н. 6 по тр.-ос. склону}
	\label{fig:IMG_3346}
\end{figure}


После обеда начали движение в \alert{14:00}. Преодолели остаток подъема по лесу и вышли к средней осыпи. По ней еще поднялись на $\sim$100м, перешли ручей и, траверсируя с небольшим набором высоты осыпь и тр.-ос. склон, вышли к небольшому озеру. Встали на ночевку вблизи озера в \alert{17:30} на плановом м.н. 6. Координаты м.н. \alert{N \degree~E \degree}.


\begin{figure}[h!]
	\centering
	\includegraphics[width=0.7\linewidth]{pics/09/camp_09} % IMG_3357.JPG
	\caption{Место ночёвки 9-10.08}
	\label{fig:camp_09}
\end{figure}

\clearpage