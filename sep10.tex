\subsection{10 сентября. пер. Кыртыкауш}
\textit{Метеоусловия: облачно, на спуске с перевала — лёгкая морось}
%\begin{figure}[h!]
%	\centering
%	\includegraphics[angle=0, width=0.7\linewidth]{pics/maps/08}
%	\label{fig:08}
%\end{figure}
\\\\
!!!!Фото на перевале: _DSC0498, _DSC0493 отсюда https://disk.360.yandex.ru/client/aa/d_kzz7CmU0WO9uMQ/photo_camera !!!!!!\\\\
Подъём в 4:00, выход в 6:40. На перевал по травянистому склону ведёт тропа, подъём простой, но долгий и под рюкзаками утомительный. Зато виды на заснеженные вершины, растворяющиеся в молочной белизне неба, и местами была брусника. В 10:25 достигли перевала Кыртыкауш, высота 3200. Спуск тоже простой по тропе. Издалека увидели убегающего в противоположную от нас сторону молодого медведя. Обед с 11:45 до 13:15, координаты (???). В 15:10 встали на место ночёвки, координаты (???). ЧХВ составило 5:51.
\\\\
textbf{Выводы и рекомендации:} пер. Кыртыкауш соответствует заявленной категории трудности н/к.  Подъём на перевал не представляет сложности. 






\clearpage
