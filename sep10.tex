\subsection{10 сентября. пер. Кыртыкауш}
\textit{Метеоусловия: облачно, на спуске с перевала~--- лёгкая морось}

\begin{figure}[h!]
	\centering
	\includegraphics[angle=0, width=0.7\linewidth]{pics/maps/10}
	\label{fig:10}
\end{figure}

\textit{Кыртыкауш~--- карач.-балк. <<Травянистый перевал>>}

\begin{figure}[h!]
	\centering
	\includegoogle[width=0.9\linewidth]{https://drive.google.com/file/d/1NCvswuzjGNbZ9yJ8dKouicaJa8L9TglX/view?usp=drive_link}
	\caption{Путь подъёма в д.р.~Исламчат}
	\label{fig:routeToKyrtyk}
\end{figure}


Подъём в 04:30, выход в 06:40. Вверх по долине идут две дороги. Новая дорога (синий цвет на рис.~\ref{fig:routeToKyrtyk}) идёт по дну долины и несколько раз пересекает реку. Старая автодорога (красный цвет на рис.~\ref{fig:routeToKyrtyk}) траверсирует левый склон реки, сейчас непроезжаема из-за сошедшей несколько лет назад сели, однако пешком проходится прекрасно. Выбираем второй вариант. 

Через некоторое время дорога выполаживается, около коша (N 43.42437° E 42.60189°) встречаем огромного алабая, лакомящегося барашком. От греха подальше приходится нести Йошту на руках (хотя, кажется, собак был неагрессивен). За левым притоком Исламсу у коша он нас покидает.

В 07:30 подсекаем спуск с пер. Каракайский Северный, снова начинается синяя маркировка (N 43.41332° E 42.59993°).

\begin{figure}[h!]
	\centering
	\includegoogle[width=0.6\linewidth]{https://drive.google.com/file/d/1gjzlNUBQzrKZc6xlT6HxkWpn9DVBkDhI/view?usp=drive_link}
	\caption{Путь подъёма в д.р.~Исламчат}
	\label{fig:routeToKyrtyk2}
\end{figure}

В 08:10 бродим левый приток р. Исламчат, который стекает с цирка пер. Исламсу (N 43.40535° E 42.60301°).

\begin{figure}[h!]
	\centering
	\includegoogle[width=0.7\linewidth]{https://drive.google.com/file/d/1MOUJ4-5YqQGRD3IHil9lzSz2TT9noRCX/view?usp=drive_link}
	\caption{Путь подъёма в д.р.~Исламчат}
	\label{fig:routeToKyrtyk3}
\end{figure}

За бродом начинается подъём на очередную моренную ступень, но тропа умудряется все-таки идти по грунту. Вообще, несмотря на высоту перевала (3400 м) нам не пришлось вообще скакать по камням, пусть и маркированным.

Выходим на седловину в 10:45. Седловина широкая, воды нет. Имеются памятные знаки в честь событий во время ВОВ. На склонах камнями выложены названия разных городов.

\begin{figure}[h!]
	\centering
	\includegoogle[width=0.7\linewidth]{https://drive.google.com/file/d/1w15UXX1Odq-bPunDxwLXVPRm3aVFluW9/view?usp=drive_link}
	\caption{Группа на пер. Кыртыкауш, вид в д.р.~Кыртык}
	\label{fig:kyrtyk1}
\end{figure}

\begin{figure}[h!]
	\centering
	\includegoogle[width=0.7\linewidth]{https://drive.google.com/file/d/1eCOHZkLzZkMFTcMU1gIUQPZs_CrGYqIt/view?usp=drive_link}
	\caption{Группа на пер. Кыртыкауш, вид в д.р.~Исламчат}
	\label{fig:kyrtyk2}
\end{figure}

На спуск с перевала ведут две тропы: маркированная по левому пхд склону и старая~--- по линии падения воды. Выбираем первый вариант, а через 150 вертикальных метров тропы сливаются. В какой-то момент видим высоко на склоне бегущего вверх медвежонка.

\begin{figure}[h!]
	\centering
	\includegoogle[width=0.7\linewidth]{https://drive.google.com/file/d/19oqH17pCps__RC5DWcw67LgLBDfWT8Nh/view?usp=drive_link}
	\caption{Спуск в д.р. Кыртык}
	\label{fig:kyrtykfall}
\end{figure}

В 11:45 доходим до брода через р.~Уллусенчи (переход простой по камням) и устраиваем обед (N 43.38576° E 42.64979°). Погода немного проясняется.

Выходим в 13:15. Дальнейшее движение не представляет проблем. За один переход доходим до пересечения с грунтовкой, ведущей в верховья левого притока Уллусенчи и по этой грунтовке в 14:00 спускаемся на дно долины р. Кыртык. По дороге окончательно, по совокупности причин, главная из которых~--- гарантированная травма колена у участника в случае затяжного спуска, отказываемся от пер. Сылтран и решаем спускаться по д.р. Кыртык. В качестве утешительного приза решаем ненадолго завернуть на скалу Уллу-Кая, посмотреть на гроты и выпить нарзану. Все это хорошо просматривается на спуске в долину (рис.~\ref{fig:kyrtyk3}).


\begin{figure}[h!]
	\centering
	\includegoogle[width=0.7\linewidth]{https://drive.google.com/file/d/1_mWzZw0MkFv4Jh9tHmhYKAgDfYJ05GRu/view?usp=drive_link}
	\caption{д.р. Кыртык, путь спуска, м.н., скала Уллу-Кая}
	\label{fig:kyrtyk3}
\end{figure}

В долине проходим мимо странного сооружения под названием <<Бивак Островок>>~--- безлюдной огороженной территории с покосившимся забором. Решаем заночевать недалеко от левого притока Кыртыка в двухстах метрах от <<Островка>>. В 15:10 встаём на ночёвку. 


ОХВ: 08:30, ЧХВ: 05:51.

Координаты м.н.: N 43.36359° E 42.68756°


\clearpage
