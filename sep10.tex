\subsection{10 сентября. пер. Кыртыкауш}
\textit{Метеоусловия: облачно, на спуске с перевала~--- лёгкая морось}

\begin{figure}[h!]
	\centering
	\includegraphics[angle=0, width=0.8\linewidth]{pics/maps/10}
	\label{fig:10}
\end{figure}


Подъём в 4:00, выход в 6:40. На перевал по травянистому склону ведёт тропа, подъём простой, но долгий и под рюкзаками утомительный. Зато виды на заснеженные вершины, растворяющиеся в молочной белизне неба, и местами была брусника. В 10:25 достигли перевала Кыртыкауш, высота 3200.

\begin{figure}[h!]
	\centering
	\includegoogle[width=0.7\linewidth]{https://drive.google.com/file/d/1w15UXX1Odq-bPunDxwLXVPRm3aVFluW9/view?usp=drive_link}
	\caption{Группа на пер. Кыртыкауш, вид в д.р.~Кыртык}
	\label{fig:kyrtyk1}
\end{figure}

\begin{figure}[h!]
	\centering
	\includegoogle[width=0.7\linewidth]{https://drive.google.com/file/d/1eCOHZkLzZkMFTcMU1gIUQPZs_CrGYqIt/view?usp=drive_link}
	\caption{Группа на пер. Кыртыкауш, вид в д.р.~Исламчат}
	\label{fig:kyrtyk2}
\end{figure}


Спуск тоже простой по тропе. Издалека увидели убегающего в противоположную от нас сторону молодого медведя. Обед с 11:45 до 13:15, координаты (???). В 15:10 встали на место ночёвки, координаты (???). ЧХВ составило 5:51.



\textbf{Выводы и рекомендации:} пер. Кыртыкауш соответствует заявленной категории трудности н/к.  Подъём на перевал не представляет сложности. 






\clearpage
