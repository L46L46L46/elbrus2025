\section{Организация и проведение похода}
\subsection{Цели и задачи маршрута. Выбор нитки маршрута}
Идеологическим вдохновителем и составителем нитки маршрута был не руководитель, а одна из участниц, Наташа Миронова. Сама идея погулять по осеннему Приэльбрусью возникла после майского н/к треккинга \cite{ostapiv2025}, когда из-за большого количества снега мы не смогли подняться ни на один перевал, равно как перевалить через юго-восточный отрог Эльбруса от водопада Терскол к водопаду Девичьи косы. Однако при детальной проработке маршрута захотелось уйти от большого количества раздробленных радиалок с логистикой между ними и пройти линейный маршрут. Первая его версия проходила через ряд перевалов 1А, таких как, например, Исламсу. Однако МКК нас предупредила, что в начале сентября на седловинах этих перевалов уже может лежать снег и, как следствие, горный поход будет считаться межсезоньем с соответствующими требованиями к опыту руководителя и участников. Поэтому и решили организовать пешеходный поход по некатегорийным перевалам. Фактически, мы <<изобрели>> стандартную нитку коммерческого маршрута из Хурзука в Верхний Баксан. Имея в ввиду вышесказанное, перечислю и побочные \textbf{цели}, которые были поставлены при планировании маршрута:

\paragraph{Посмотреть красивое.} Изюминкой маршрута стал вид на Эльбрус с нескольких сторон, водопады в Джилы-Су и пещеры горы Улkукая.

\paragraph{Погулять по невысоким горам.} Руководителю похода не доводилось раньше проводить поход по среднегорью, с его особенным ландшафтом~--- исправляем!
	
	
\paragraph{Сходить в горы осенью.} Возможность посмотреть на то, что такое осень в горах стала немаловажным фактором при выборе времени похода. К тому же мы надеялись на сухую ссентябрьскую погодав. Тем не менее, оказалось, что осень~--- это плачущее небе под ногами$\ldots$

\paragraph{Не упарываться.} Хотелось избежать технически сложных участков на маршруте, поскольку в этом сезоне для таких целей у нас был Тескей \cite{teskei2025ostapiv}, а также потому что некоторые участники группы не обладают достаточной квалификацией для преодоления их в межсезонье. Маршрут даёт возможность пройти простыми треккинговыми тропами.

В выборе района, конечно, огромную роль играла и простота логистики.

\subsection{Логистика}
Подьезд первой половины группы осуществлялся на поезде Москва---Кисловодск до станции Минеральные Воды (прибытие в 7:20). Оставшаяся половина группы добиралась до места на самолете Уральских авиалиний Москва---Минеральные Воды, рейс U6-153 (прибытие в 14:45). От Минеральных Вод до а. Хурзук добирались на трансфере, заказанном через Бориса Саракуева (+7(928)-950-38-68, +7(929)-884-31-75, bezonec@list.ru). Время движения~--- порядка 4 часов. Стоимость трансфера туда составила 12000~\faRub.~Обратно до Минеральных Вод добирались на трансфере, так же заказанном через Бориса Саракуева, из Верхнего Баксана. Время движения~--- 3 часа, стоимость~--- 10000~\faRub.

\subsection{Аварийные выходы из маршрута и его запасные варианты}
\textbf{Аварийными выходами} с маршрута являлось следование по его нитке до ближайшей цивилизации. В первые два дня таковым является движение обратно в сторону Хурзука, далее, от пер. Чемарт и до пер. Кыртыкауш~--- спуск в Джилы-Су, а далее~--- спуск в Верхний Баксан по д.р. Кыртык.


\textbf{Запасными вариантами} маршрута являлся отказ от прохождения пер. Сылтран и спуск по д.р. Кыртык в Верхний Баксан.


\subsection{Изменение маршрута и их причины}
Воспользовались запасным вариантом~--- отказ от прохождения пер. Сылтран и спуск в Верхний Баксан по д.р. Кыртык. Сделали это в связи с тем, что длительный спуск (около 2 км по вертикали) почти гарантированно привёл бы к проблемам с коленями у одного из участников. К тому же, судя по отзывам встреченных нами туристов, на седловине перевала уже лежал снег.

\subsection{Обеспечение безопасности на маршруте}
Группа была зарегистрирована в МЧС по КЧР и КБР (две заявки, отправленные за 4 дня до выхода на маршрут). На маршруте дежурные МЧС по КЧР просили предоставлять всю информацию о передвижении (старт, пересечение границы республик, финиш), причём, после того, как отзвонились из КБР, перевели нашу заявку в местное МЧС. Дежурный по КБР, наоборот, не просил отзваниваться, мотивируя тем, что необходимые даты указаны в заявке (а на финише удивлялся, откуда взялись две заявки; видимо, это произошло благодаря ретивым сотрудникам из КЧР). Номера дежурных: для КЧР: +7(878) 226-62-00, для КБР: +7(8662)742-556.

Для обмена сообщениями, отслеживания положения группы на карте, а также возможности экстренной связи, в группе имелся спутниковый треккер IRIDIUM Rockstar 360. Стоимость аренды треккера в компании <<Satellite-Rent>> составила 600~\faRub~в день, залог~--- 50000~\faRub, за услуги связи~--- 3780~\faRuble,~70~\faRub~за юнит. К сожалению, связь была плохой, точки не отправлялись, а сообщения смогли отправиться только несколько раз за поход. Тем не менее, наличие трекера было оправдано, ибо все-таки известия доходили и, что главное, из д.р. Кыртык смогли отправить координатору сообщение с просьбой перенести трансфер на два дня раньше, и к моменту нашего выхода на связь транспорт уже был организован.

Каждый участник самостоятельно оформлял на себя индивилуальный страховой полис. Выбрали страховую фирму <<Совкомбанк страхование>>, ассист AP Companies, размер страховой защиты 35000~USD,  вид отдыха <<Экстремальный отдых>>. Стоимость полиса составила 2012~\faRuble~с человека.

\subsection{Перечень наиболее интересных природных и исторических объектов, занятий на маршруте}
\begin{enumerate}[noitemsep,topsep=0pt,parsep=0pt,partopsep=0pt]
	\item Хребет Енукол~--- высокогорные степи;
	\item Плато Ирахитсырт в Северном Приэльбрусье;
	\item Джилы-Су с его нарзанами, термальными источниками, Калиновым мостом и водопадо Султан;
	\item Скала Уллу-Кая в д.р. Кыртык (каменоломни, потрясающего видак скалы, нарзан).
\end{enumerate}

\paragraph{Темы практических занятий:}

\begin{itemize}
	\item Техника передвижения по травянистым и осыпным склонам;
	\item Техника несложных бродов поодиночке;
\end{itemize}

\newpage
\subsection{Развёрнутый график движения}
\alert{Артур, это на тебе, пожалуйста}
\begin{table}[h!]
	\centering
	\resizebox{0.95\textwidth}{!}{%
		\begin{tabular}{|>{\centering\arraybackslash}m{0.045\linewidth}
				|>{\centering\arraybackslash}m{0.02\linewidth}
				|>{\centering\arraybackslash}m{0.43\linewidth}
				|>{\centering\arraybackslash}m{0.09\linewidth}
				|>{\centering\arraybackslash}m{0.1\linewidth}
				|>{\centering\arraybackslash}m{0.05\linewidth}
				|>{\centering\arraybackslash}m{0.09\linewidth}
				|>{\centering\arraybackslash}m{0.13\linewidth}|}
			\hline						
			Дата	&	\begin{turn}{90}День\end{turn}	&	Участок маршрута	&	Км с $k=1.2$	&	Набор /сброс, м	&	ЧХВ	&	Высота ночёвки, м	&	Способы передвижения	\\
			\hline
			
			18.08	&	1	&	г.~Минеральные воды~--- аул Верхний Учкулан~--- д.р Учкулан~--- д.р. Кичкинакол Уллукёльский	&	5.3	&	$+650$\newline$-0$	& 2:46	&	2200	&	Машина,\newline Пешком	\\
			\hline
			19.08	&	2	&	д.р. Кичкинакол Уллукёльский~--- оз. Гитче-Кёль~--- оз. Уллу-Кёль 	&	5.6	& $+650$\newline$-0$		& 3:25		& 2850		&	Пешком	\\
			\hline
			20.08	&	3	&	м.н.~--- \textbf{пер. Уллу-Кёль Восточный (1А$^\star$, 3050)}~--- кош в д.р. Трёхозёрная~--- д.р. Махар	&	7.2	& $+200$\newline$-1190$		& 7:39	& 1860		&	Пешком	\\
			\hline
			21.08	&	4	&	м.н.~--- т/б <<Глобус>>~--- д.р. Гондарай~--- д.р. Джалпаккол	&	11.3	&$+390$\newline$-225$		& 3:54		& 2120		&	Пешком	\\
			\hline
			22.08	&	5	&	м.н.~--- д.р. Кичкинекол Джалпаккольский~--- м.н. под моренным валом пер. Джалпаккол Северный	&	5.8	& $+620$\newline$-0$		& 3:56	& 2740		&	Пешком	\\
			\hline
			23.08	&	6	&	м.н.~--- \textbf{пер. Джалпаккол Северный (1А$^\star$, 3411)}~--- зелёные ночёвки на спуске в д.р. Мырды	&	5.0 	& $+660$\newline$-395$		& 6:16		& 3015		&	Пешком	\\
			\hline
			24.08	&	7	&	м.н.~--- д.р. Мырды~--- а/л <<Узункол>>	&	7.5	& $+0$\newline$-960$		& 3:53		& 2060		&	Пешком	\\
			\hline
			25.08	&	8	&	м.н.~--- д.р. Кичкинекол~--- д.р. Таллычат~--- Поляна Крокусов	&	7.1	& $+780$\newline$-0$		& 3:23		& 2840		&	Пешком	\\
			\hline
			26.08	&	9	&	м.н.~--- \textbf{пер. Кичкинекол Малый (1А, 3206)}~--- д.р. Чунгур-Джар	&	4.6	& $+360$\newline$-520$		& 2:42		& 2680		&	Пешком	\\
			\hline
			27.08	&	10	&	м.н.~--- \textbf{пер. Перемётный (1А, 3255)}~--- д.р. Танышхан	&	7.1	& $+575$\newline$-935$		& 6:50		& 2320		&	Пешком	\\
			\hline
			28.08	&	11	&	м.н.~--- д.р. Чиринкол~--- д.р. Кубань &	12.7	& $+90$\newline$-500$		& 3:23		& 1890		&	Пешком	\\
			\hline
			29.08	&	12	&	м.н.~--- погранзастава <<Хурзук>>~(рад.)~--- д.р. Уллу-Кам	&	20.9	& $+1210$\newline$-370$		& 7:15		& 2725		&	Пешком	\\
			\hline
			30.08	&	13	&	м.н.~--- \textbf{пер. Хотютау (1А$^\star$, 3546)}~--- лед. Большой Азау~--- оз. Эльбрусское~--- ст. <<Старый Кругозор>>~--- поляна Азау & 10.9	& $+800$\newline$-615$		& 4:25		& 2915		&	Пешком, Канатная дорога	\\
			\hline
			\multicolumn{3}{|c|}{\textbf{\textit{\Large{Итого:}}}} & \large{\textbf{111.0}} & \large{$\mathbf{+6985}$\newline$\mathbf{-5210}$	}	& \multicolumn{3}{c|}{\large{\textbf{58:08}\newline\textbf{2д 10ч 08мин}}} \\
			\hline
		\end{tabular}
}	
	
\end{table}



\clearpage