\section{Организация и проведение похода}
\subsection{Цели и задачи маршрута. Выбор нитки маршрута}
Идеологическим вдохновителем и составителем нитки маршрута был не руководитель, а одна из участниц, Наташа Миронова. Сама идея погулять по осеннему Приэльбрусью возникла после майского н/к треккинга \cite{ostapiv2025}, когда из-за большого количества снега мы не смогли подняться ни на один перевал, равно как перевалить через юго-восточный отрог Эльбруса от водопада Терскол к водопаду Девичьи косы. Однако при детальной проработке маршрута захотелось уйти от большого количества раздробленных радиалок с логистикой между ними и пройти линейный маршрут. Первая его версия проходила через ряд перевалов 1А, таких как, например, Исламсу. Однако МКК нас предупредила, что в начале сентября на седловинах этих перевалов уже может лежать снег и, как следствие, горный поход будет считаться межсезоньем с соответствующими требованиями к опыту руководителя и участников. Поэтому и решили организовать пешеходный поход по некатегорийным перевалам. Фактически, мы <<изобрели>> стандартную нитку коммерческого маршрута из Хурзука в Верхний Баксан \cite{kommersy}, хотя категорийные походы в этих краях тоже есть \cite{elbrus2022sevlova}. Имея в ввиду вышесказанное, перечислю и побочные \textbf{цели}, которые были поставлены при планировании маршрута:

\paragraph{Посмотреть красивое.} Изюминкой маршрута стал вид на Эльбрус с нескольких сторон, водопады в Джилы-Су и пещеры горы Уллукая.

\paragraph{Погулять по невысоким горам.} Руководителю похода не доводилось раньше проводить поход по среднегорью, с его особенным ландшафтом~--- исправляем!
	
	
\paragraph{Сходить в горы осенью.} Возможность посмотреть на то, что такое осень в горах стала немаловажным фактором при выборе времени похода. К тому же мы надеялись на сухую сентябрьскую погоду. Тем не менее, оказалось, что осень~--- это плачущее небе под ногами$\ldots$

\paragraph{Не упарываться.} Хотелось избежать технически сложных участков на маршруте, поскольку в этом сезоне для таких целей у нас был Тескей \cite{teskei2025ostapiv}, а также потому что некоторые участники группы не обладают достаточной квалификацией для преодоления их в межсезонье. Маршрут даёт возможность пройти простыми треккинговыми тропами.

В выборе района, конечно, огромную роль играла и простота логистики.

\subsection{Логистика}
Подъезд первой половины группы осуществлялся на поезде Москва---Кисловодск до станции Минеральные Воды (прибытие в 7:20). Оставшаяся половина группы добиралась до места на самолете Уральских авиалиний Москва---Минеральные Воды, рейс U6-153 (прибытие в 14:45). От Минеральных Вод до а. Хурзук добирались на трансфере, заказанном через Бориса Саракуева (+7(928)-950-38-68, +7(929)-884-31-75, bezonec@list.ru). Снова в восторге от качества оказания услуг~--- Борис смог по нашей просьбе, переданной через координатора, оперативно перенести обратный трансфер на день раньше. Время движения~--- порядка 4 часов. Стоимость трансфера туда составила 12000~\faRub.~Обратно до Минеральных Вод добирались на трансфере, так же заказанном через Бориса Саракуева, из Верхнего Баксана. Время движения~--- 3 часа, стоимость~--- 10000~\faRub.

\subsection{Аварийные выходы из маршрута и его запасные варианты}
\textbf{Аварийными выходами} с маршрута являлось следование по его нитке до ближайшей цивилизации. В первые два дня таковым является движение обратно в сторону Хурзука, далее, от пер. Чемарт и до пер. Кыртыкауш~--- спуск в Джилы-Су, а далее~--- спуск в Верхний Баксан по д.р. Кыртык.


\textbf{Запасными вариантами} маршрута являлся отказ от прохождения пер. Сылтран и спуск по д.р. Кыртык в Верхний Баксан.


\subsection{Изменение маршрута и их причины}
Воспользовались запасным вариантом~--- отказ от прохождения пер. Сылтран и спуск в Верхний Баксан по д.р. Кыртык. Сделали это в связи с тем, что длительный спуск (около 2 км по вертикали) почти гарантированно привёл бы к проблемам с коленями у одного из участников. К тому же, судя по отзывам встреченных нами туристов, на седловине перевала уже лежал снег.

\subsection{Обеспечение безопасности на маршруте}
Группа была зарегистрирована в МЧС по КЧР и КБР (две заявки, отправленные за 4 дня до выхода на маршрут). На маршруте дежурные МЧС по КЧР просили предоставлять всю информацию о передвижении (старт, пересечение границы республик, финиш), причём, после того, как отзвонились из КБР, перевели нашу заявку в местное МЧС. Дежурный по КБР, наоборот, не просил отзваниваться, мотивируя тем, что необходимые даты указаны в заявке (а на финише удивлялся, откуда взялись две заявки; видимо, это произошло благодаря ретивым сотрудникам из КЧР). Номера дежурных: для КЧР: +7(878) 226-62-00, для КБР: +7(8662)742-556.

Для обмена сообщениями, отслеживания положения группы на карте, а также возможности экстренной связи, в группе имелся спутниковый треккер IRIDIUM Rockstar 360. Стоимость аренды трекера в компании <<Satellite-Rent>> составила 600~\faRub~в день, залог~--- 50000~\faRub, за услуги связи~--- 3780~\faRuble,~70~\faRub~за юнит. К сожалению, связь была плохой, точки не отправлялись, а сообщения смогли отправиться только несколько раз за поход. Тем не менее, наличие трекера было оправдано, ибо все-таки известия доходили и, что главное, из д.р. Кыртык смогли отправить координатору сообщение с просьбой перенести трансфер на два дня раньше, и к моменту нашего выхода на связь транспорт уже был организован.

Каждый участник самостоятельно оформлял на себя индивидуальный страховой полис. Выбрали страховую фирму <<Совкомбанк страхование>>, ассист AP Companies, размер страховой защиты 35000~USD,  вид отдыха <<Экстремальный отдых>>. Стоимость полиса составила 2012~\faRuble~с человека.

\subsection{Перечень наиболее интересных природных и исторических объектов, занятий на маршруте}
\begin{enumerate}[noitemsep,topsep=0pt,parsep=0pt,partopsep=0pt]
	\item Хребет Енукол~--- высокогорные степи;
	\item Эльбрус, который в этом походе можно было рассмотреть с трёх сторон~--- с запада, севера и востока;
	\item Плато Ирахитсырт в Северном Приэльбрусье;
	\item Джилы-Су с его нарзанами, термальными источниками, Калиновым мостом и водопадом Султан;
	\item Скала Уллу-Кая в д.р. Кыртык (каменоломни, потрясающего вида скалы, нарзан).
\end{enumerate}

\paragraph{Темы практических занятий:}

\begin{itemize}
	\item Техника передвижения по травянистым и осыпным склонам;
	\item Техника несложных бродов поодиночке;
\end{itemize}

\newpage
\subsection{Развёрнутый график движения}

\begin{table}[h!]
	\centering
	\caption{Хронометраж маршрута}
	\begin{adjustbox}{max width=\textwidth}
		\begin{tabular}{|c|c|>{\centering\arraybackslash}m{0.3\linewidth}|c|c|>{\centering\arraybackslash}m{0.13\linewidth}|>{\centering\arraybackslash}m{0.1\linewidth}|c|>{\centering\arraybackslash}m{0.15\linewidth}|}
			\hline
			\textbf{Дата} & \textbf{День} & \textbf{Участок маршрута} & \textbf{Км} & \textbf{ЧХВ} & \textbf{Набор Сброс(м)} & \textbf{Высота ночевки, (м)} & \textbf{ОХВ} & \textbf{min/max высота(м)} \\
			\hline
			05.09.25 & 1 & Хурзук --- д.р. Уллу-Хурзук & 2.7 & 0:46 & +127 & 1627 & 0:46 & 1500/1627 \\
			\hline
			06.09.25 & 2 & д.р. Еникол --- \textbf{пер. Еникол (н/к, 2588)} --- д.р. Эльмезтебе & 13.5 & 5:56 & +1241 -248 & 2816 & 9:30 & 1627/2868 \\
			\hline
			07.09.25 & 3 & \textbf{Пер. Быкылы (н/к, 2963)} --- д.р. Перевальная --- \textbf{пер. Чемарт (н/к, 3137)} --- д.р. Чемарткол & 12.6 & 3:06 & +689 -813 & 2682 & 5:12 & 2682/3133 \\
			\hline
			08.09.25 & 4 & \textbf{Пер. Бурунташ (н/к, 3072)} --- \textbf{брод р. Кызылкол (н/к)} --- д.р. Кызылкол --- а/л Лакколит & 11.7 & 4:29 & +575 -682 & 2576 & 6:15 & 2575/3084 \\
			\hline
			09.09.25 & 5 & Д.р. Кызылкол --- Джилы-Су --- д.р. Малка --- \textbf{пер. Бересун (н/к, 3072)} --- д.р. Шаукам & 14.3 & 5:06 & +386 -683 & 2280 & 5:25 & 2248/2576 \\
			\hline
			10.09.25 & 6 & д.р. Исламчат --- \textbf{пер. Кыртыкауш (н/к, 3242)} --- д.р. Уллуусенчи --- д.р. Кыртык & 16.4 & 5:51 & +1043 -960 & 2372 & 8:30 & 2275/3246 \\
			\hline
			11.09.25 & 7 & Д.р. Кыртык & 12.7 & 4:34 & +376 -1191 & 1572 & 5:15 & 2387/1572 \\
			\hline
		\end{tabular}
	\end{adjustbox}
\end{table}

\clearpage