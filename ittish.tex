\subsection{Пер. Иттиш}

Пер. Иттиш ориентирован с севера на юг и является перевалом через основной хр. Тескей-Ала-Тоо. Это один из немногих перевалов (наряду с пер. Кашкасу, Джукучак) через данный основной хребет, которые имеют (низкую) трудность 1А. Вообще говоря, пер. Иттиш~--- это выход на горизонтальное плато, на котором расположены сырты (заболоченные луга). Лишь северная сторона перевала имеет наклон. Перевал чаще проходится с юга на север \cite{kovinov2021,sergeev2024,tipsina2024}; в данном походе было решено преодолеть его на подъём с севера на юг. При подготовке использовались отчёты \cite{kovinov2021,sergeev2024,tipsina2024}, также о возможности проходения перевала в обратную сторону руководитель узнавал у Натальи и Григория по почте.

Подъём на перевал можно разбить на две части: подъём по тропе по хвойному лесу и осыпи и горизонтальное движение по средней и крупной осыпи. Второй этап недлинный по расстоянию~--- около 4--5 км~--- но может занять много времени из-за трудности рельефа. В некоторых местах можно сойти с крупных камней и двигаться по пересохшему дну озер, находящихся вблизи осыпи. При подходе к седловине видно обледенелый пик Иттиш и ледник, примыкающий к нему (цирк пер. Иттиш Левый (2А)). После взятия перевала спуска нет~--- сразу идёт выход на высокогорные сырты и озёра. В хорошую погоду с седловины открывается вид на соседний хр. Акшийрак. В целом, подъём является постепенным и не имеет крутых взлётов.

Суммарный перепад высот от д.р. Джууку, из которой начинался подъем, до седловины перевала составляет около 1000Ём, поэтому было решено устроить ночёвку в середине подъёма на небольшом озере, находящимся выше границы леса на высоте 3300~м. От м.н. на озере до седловины перевала был пройден остаток подъёма по средней осыпи и затем преодолен упомянутый затяжной участок горизонтальной крупной осыпи.