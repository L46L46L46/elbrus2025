\subsection{12 августа. Пер. Кашкасу (1A)}
\textit{Метеоусловия: Весь день переменная облачность; к 16.50 распогодилось, ясно.}

\begin{figure}[h!]
	\centering
	\includegraphics[angle=0, width=0.5\linewidth]{pics/maps/12}
	\label{fig:12}
\end{figure}

Подъём дежурных в 4:30, в 05:40 группа (кроме Маши) выходит в бодром темпе на в. Марс. Подъём на вершину прост. От лагеря подходим по долине под склон (1 км, около 20 мин ЧХВ), далее поднимаемся по широкому кулуару~--- руслу ручья (можно набрать воды) в амфитеатр, что занимает ещё около 45 мин ЧХВ. Далее поднимаемся на сам амфитеатр и идём по предвершинному плато к вершине (30 мин ЧХВ). В 07:18 выходим на вершину и устраиваем получасовой привал. Погода великолепная, виды потрясающие.


\begin{figure}[h!]
	\centering
	\includegraphics[width=0.7\linewidth]{pics/12/mars_raise.jpg}
	\caption{Слева: подъём по кулуару, справа: движение по предвершинному плато}
	\label{fig:mars_raise.jpg}
\end{figure}

\begin{figure}[h!]
	\centering
	\includegraphics[width=0.7\linewidth]{pics/12/IMG_3590.jpg}
	\caption{Группа на в. Марс (н/к, 4350)}
	\label{fig:IMG_3590.jpg}
\end{figure}


К 9.15 возвращается, группа обедает и в 10.58 начинает движение вдоль левого берега системы озёр. 15 мин, с 11.15 по 11.30, ушло на брод ручья,
5 и 10 мин — на привалы в 12.00 и 13.05 соответственно. В 13.35 достигли седловой точки пер. Кашкасу. В 14.10 начали спуск по гребням 
моренных валов. Дважды устраивали привал: 5-минутный в 15.00 и 12-миминутный в 15.55. В 16.50 встали на ужин и впоследствии ночёвку в точке, обозначенной на треке как 
"обед классное место". 
\\\\ ЧХВ составило 7 ч 48 мин.


\begin{figure}[h!]
	\centering
		\includegraphics[width=0.7\linewidth]{pics/10/camp_10}
	\caption{Место ночёвки 10-11.08 — \textcolor{red}{заменить!!!!!}}
	\label{fig:camp_10}
\end{figure}

\clearpage
