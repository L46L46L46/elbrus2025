\subsection{11 сентября. Долина р.Кыртык, финиш}
\textit{Метеоусловия: утром туман, к полудню развиднелось, после солнечно.}

\begin{figure}[h!]
	\centering
	\includegraphics[angle=90, width=0.65\linewidth]{pics/maps/11}
	\label{fig:11}
\end{figure}

\textit{Уллу-Кая}~--- карач.-балк. <<Большая скала>>.

Подъём в 05:30. В 07:10 мы, позавтракамши да постоявши в планке, вместе с клочьями тумана поползли вниз по долине р. Кыртык, движение по грунтовой дороге сложности не представляло. В качестве утешительного приза решили заглянуть в расположенную в долине скалу Уллу-Кая. К ней от коша (N 43.35954° E 42.69623°) ведет тропа. Сразу над кошем~--- большой камень, Кидаем там рюкзаки и в 07:30 идём наверх. Можно выпить нарзана и полюбоваться рукотворными(?) пещерами и пещерками. Местами крутизна склона позволяет устроить мини-тренировку по самостраховке альпешнтоком.

\begin{figure}[h!]
	\centering
	\includegoogle[width=0.9\linewidth]{https://drive.google.com/file/d/1Y8UcewaWIQxXiR-wuQorq8z-W2FgD6RM/view?usp=drive_link}
	\caption{Вид на Уллу-Кая}
	\label{fig:uulu-kaya}
\end{figure}

\begin{figure}[h!]
	\centering
	\includegoogle[width=0.9\linewidth]{https://drive.google.com/file/d/1HxpuhSZbKa2PF9ybUVE3lEnrclrws4k0/view?usp=drive_link}
	\caption{Уллу-Кая, пещеры}
	\label{fig:uulu-kaya1}
\end{figure}

В 09:33 возвращаемся на дно д.р. Кыртык и продолжаем движение вниз. Нам предстоит десятикилометровый забег по хорошей грунтовке. Дорога приятная, располагает к неспешным размышлениям. 

\begin{figure}[h!]
	\centering
	\includegoogle[width=0.9\linewidth]{https://drive.google.com/file/d/1eoEv-247uQLUHF_MoWEojWYyTCCaLARd/view?usp=drive_link}
	\caption{\textit{Нас кличе у мандри дорога...}}
	\label{fig:doroga}
\end{figure}


С толку сбивают разве что столбики-указатели расстояний до Верхнего Баксана и до перевала Сылтран, которые явно стоят не по порядку (а некоторые~--- ещё и задом наперёд)

\begin{figure}[h!]
	\centering
	\includegoogle[width=0.8\linewidth]{https://drive.google.com/file/d/1aXT8NLjN4geCvIPdxU1re7VCFDyjTdH_/view?usp=drive_link}
	\caption{Д.р. Кыртык и неинформативные столбики (справа)}
	\label{fig:kyrtyk}
\end{figure}

Встречаем группу из четырёх пермский туристов, которые идут горный поход 3 к.с. и сейчас встают на бивак. Продолжаем движение, и примерно через километр встречаем ожидающих их товарищей. Оказывается, они (ожидающие) перепутали то ли направление течения реки, то ли саму реку и ждали первую группу, хотя всем нужно было в сторону Эльбруса, кажется, на перевал Российских офицеров (2А). В общем, наше пребывание в этих краях оказалось небесполезным и, как минимум, сэкономило пермским туристов некоторое количество времени \smiley.


Как только в прямой видимости оказывается Верхний Баксан, начинает ловить мобильная связь (N 43.33365° E 42.75946°). Пользуемся этой возможностью и с большой радостью узнаём, что Дима по нашей просьбе перенёс трансфер на сегодня, и нам не придётся лишний день задерживаться. 

\begin{figure}[h!]
	\centering
	\includegoogle[width=0.7\linewidth]{https://drive.google.com/file/d/1UpJzQFUO2GCOQ5IAHviDfgG63-GlZEef/view?usp=drive_link}
	\caption{Группа перед а. Верхний Баксан. До финиша 1 км}
	\label{fig:finish}
\end{figure}

На финальном участке тропы (обход каньона р. Кыртык, сброс около 300 вертикальных метров) открывается прекрасный вид на аул, долину Баксана, подъём на перевал Сылтран.

\begin{figure}[h!]
	\centering
	\includegoogle[width=0.7\linewidth]{https://drive.google.com/file/d/1JuGjKrhDQz8teDoPwkWv1lGmoT96ktx-/view?usp=drive_link}
	\caption{Финальный участок спуска}
	\label{fig:baksan}
\end{figure}

В 12:50 спускаемся в аул, перепаковываемся. В 14:00 нас забирает трансфер, а в 18:00 уже возвращаемся на поезде кто куда.

\clearpage
