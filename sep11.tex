\subsection{11 сентября. Долина р.Кыртык, финиш}
\textit{Метеоусловия: утром туман, к полудню развиднелось, после солнечно.}

%\begin{figure}[h!]
%	\centering
%	\includegraphics[width=0.7\linewidth]{pics/09/camp_09} % IMG_3357.JPG
%	\caption{Место ночёвки 09--10.08}
%	\label{fig:camp_09}
%\end{figure}

Подъём в 4:30. В 7:10 мы, позавтракамши да постоявши в планке, вместе с клочьями тумана поползли вниз по долине, движение по грунтовой дороге сложности не представляло. Через 1:30 ЧХВ бросили рюкзаки у выхода в ущелье с пещерами и налегке полезли вверх. Тропу мы в самом начале потеряли, т.к. в какой-то момент забрали левее, и какое-то время двигались параллельно ей. Таким образом мы создали себе с технической точки зрения самый сложный и интересный участок маршрута: крутой травянисто-осыпной склон без выраженной тропы. Затем полезли перпендикулярно тропе наверх и оказались над нижними пещерами. Там тоже были пещеры с обильными следами жизнедеятельности горных травоядных. Вернулись тем же путём, но уже найдя на тропу, мимо нарзана, и в 9:15 возобновили движение вниз по долине и в 14:00 были уже на вокзале в Минеральных водах. 

\clearpage
