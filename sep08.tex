\subsection{08 августа. пер. Бурунташ, а/л <<Лакколит>>}
\textit{Метеоусловия: ночью дождь, утром переменная облачность, ясно, днём, вечером, ночью дождь}

\begin{figure}[h!]
	\centering
	\includegraphics[angle=90, width=0.8\linewidth]{pics/maps/08}
	\label{fig:08}
\end{figure}

Начиная с 04:00 сквозь сон мониторим обстановочку. Встаём в 05:00, как только прекращается дождь. Завтракаем, любуемся свежим снегом на вершинах (запорошило склоны выше приблизительно 3500~м). Выходим в 06:25 и движемся по тропе вдоль правого притока р. Чемарткол в сторону пер.~Бурунташ. Кажется, что в этих краях коровки--биологические газонокосилки не ходят, поэтому тропа норовит скрыться в мокрой траве (рис.~\ref{fig:burun_river}).

\begin{figure}[h!]
	\centering
	\includegoogle[width=0.8\linewidth]{https://drive.google.com/file/d/1qrRYZxMfuf8lq3XnpOfDnIqa0ttqPWRv/view?usp=drive_link}
	\caption{Подъём на пер. Бурунташ}
	\label{fig:burun_river}
\end{figure}

Затем тропа заворачивает налево пхд, обходя небольшой прижим. Здесь на протяжении 200 горизонтальных метров даже есть ощутимый уклон до 20\degree. 

\begin{figure}[h!]
	\centering
	\includegoogle[width=0.7\linewidth]{https://drive.google.com/file/d/1e0Niv_SpAj9JCnEIDDpLY4pJaek7L2Nt/view?usp=drive_link}
	\caption{Подъём на пер. Бурунташ}
	\label{fig:burun_uphill}
\end{figure}


Однако далее, чем ближе к седловине, тем более тропа напоминает скоростное шоссе, идущее, причём, по обоим берегам ручья. В целом, тропа на пер. Бурунташ показалась достаточно интересной именно из-за своего разнообразия.

\begin{figure}[h!]
	\centering
	\includegoogle[width=0.7\linewidth]{https://drive.google.com/file/d/1ZbswaeJXmXBv1cY8FkzUNN5LK-6zX6Wb/view?usp=drive_link}
	\caption{Подъём на пер. Бурунташ}
	\label{fig:burun_uphill_1}
\end{figure}

По дороге КООРДИНАТЫ видим несколько мест для ночевки. Воды много.


На седловину пер. Бурунташ выходим в 08:35. Она представляет собой очень широкое плато, на котором, поначалу, даже непонятно, где искать перевальный тур. Находим его около большого камня с памятной табличкой, установленной в 2025 году в честь годовщины победы в ВОВ, координаты: N 43.42390° E 42.42623°. К сожалению, вокруг много лже-туриков с мусором внутри :(. Погода ясная, вид на Эльбрус и окрестности потрясающий. Устраиваем фотосессию на фоне горы.

\begin{figure}[h!]
	\centering
	\includegoogle[width=0.7\linewidth]{https://drive.google.com/file/d/1uoUO_yBKoYSdvhsErMBe_y_Mrm1I_joE/view?usp=drive_link}
	\caption{пер. Бурунташ, вид на Эльбрус}
	\label{fig:buruntash1}
\end{figure}

\begin{figure}[h!]
	\centering
	\includegoogle[width=0.7\linewidth]{https://drive.google.com/file/d/1YDXBwDvdmo66RN5_xplST0r_Vy_PMBWZ/view?usp=drive_link}
	\caption{пер. Бурунташ, вид в долину правого притока Чемарткола (путь подъёма)}
	\label{fig:buruntash2}
\end{figure}

В 9:15 начинаем спуск. С восточной стороны склон также пологий и по нему идут несколько троп. По одной из них в 10:03  спускаемся к р.~Кызылкол (карач.-балк. \textit{<<Красное ущелье>>}). Мы знаем, что на точке N 43.42654° E 42.43766° должен быть натянут трос для переправы, но знаем также, что, судя по отзывам туристов этого сезона, в количестве присутствуют места, в которых реку можно перейти по камням. Одно из таких мест находим здесь: N 43.42483° E 42.43703°. Сомнения вызывает только шаг с предпоследнего камня на последний. На всякий случай делаем этот шаг без рюкзаков, которые предварительно бросаем на целевой берег (речь идет о расстоянии около метра, так что упустить рюкзак в воду было бы сложно). На всё про всё у наш ушло меньше 7 минут.

\begin{figure}[h!]
	\centering
	\includegoogle[width=0.9\linewidth]{https://drive.google.com/file/d/1MtaAatVQGYh2Eem3x5S_7JmeE5qXaUbz/view?usp=drive_link}
	\caption{Слева: самцы в напряге (передача более длинных палок); по центру и справа: переправа последнего участника}
	\label{fig:pereprava}
\end{figure}

Далее нам необходимо забраться на плато Ирахиксырт. Для этого можно двигаться по тропе, спустившись вдоль реки на 40 вертикальных метров, а затем набрав около 100 м по склону. Мы хотим сэкономить паразитный сброс и движемся траверсом по склону в надежде в дальнейшем подрезать тропу. Однако надежды оказались напрасны: мы вырулили прямо на небольшой распадок и всё-таки были вынуждены организовать паразитный сброс. Зато потренировали спуск по травянистому склону с использованием альпенштока и движение плотной группой.

\begin{figure}[h!]
	\centering
	\includegoogle[width=0.4\linewidth]{https://drive.google.com/file/d/1lBvIVKY6-MzG7B5uPtWJXVRbOdTQjTe0/view?usp=drive_link}
	\caption{Тренировка спуска по травянистому склону с использованием альпенштока и движения плотной группой}
	\label{fig:trenya}
\end{figure}



Так или иначе, к 11:00 таки выходим на плато, любуемся открывшимися видами и интересными минеральными выходами на склонах левого берега р.~Кызылкол.

\begin{figure}[h!]
	\centering
	\includegoogle[width=0.7\linewidth]{https://drive.google.com/file/d/1KizZE64GMWdahdubWcXPDyvJi-6d59Ps/view?usp=drive_link}
	\caption{Панорама плато Ирахиксырт, вид на восток}
	\label{fig:plauteau}
\end{figure}

По плато идёт отличная тропа, без проблем пересекаем его и подходим к м.н.~--- а/л <<Лакколит>>. Перед спуском с плато ловим связь и мобильный интернет (редкость в наше время!) у оператора <<Мегафона>> (и у <<Tele2>>, который поймал вышку первого).

\begin{figure}[h!]
	\centering
	\includegoogle[width=0.7\linewidth]{https://drive.google.com/file/d/1Sq8n4ApW8FN0YOp8Ivcp2r6GiudJT4-B/view?usp=drive_link}
	\caption{Спуск с плато, вид в сторону Джилы-Су}
	\label{fig:lakkolit}
\end{figure}

\begin{figure}[h!]
	\centering
	\includegoogle[width=0.9\linewidth]{\url{https://drive.google.com/file/d/1cYX3GR8wBCrL91uHwuEJXRGMFmWCdKg9/view?usp=drive_link}}
	\caption{Спуск с плато, вид в сторону плато Ирахиксырт. Фото сделано утром следующего дня}
	\label{fig:lakkolit2}
\end{figure}

В 13:00 спускаемся к альплагерю и встаём лагерем. За 500~\faRub~с человека в нашем распоряжении общественная палатка со столами и скамьями для готовки еды и уборная. Горячий душ стоит 400~\faRub~с человека.

ЧХВ: 4:30, ОХВ: 6:15

\clearpage