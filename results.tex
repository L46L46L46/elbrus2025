\section{Итоги похода, выводы и рекомендации по совершённому походу}

	\begin{enumerate}
		\item Одной из причин выбора сентября как месяца проведения похода была уверенность, что начало сентября не сильно отличается от конца августа в рамках погодных условий. Ожидалось, что будет сухо, солнечно и не дождливо. Тем не менее, во время похода часто шёл дождь, а примерно 7 сентября выше 3500 м появился снег. Это никак не повлияло на преодоление группой маршрута, но лишний раз подтвердило мнение, высказанное МКК, что даты похода уже стоит считать межсезоньем; % Не уверена, что это вывод, но упомянуть хочется.
		\item При планировании маршрута можно было закладывать гораздо большее расстояние, которое группа могла бы пройти за день. Особенно это чувствовалось в самом начале маршрута, когда движение осуществлялось по технически несложному хребту Енукол;
		\item Сам отказ от перевала Сылтран руководитель считает правильным решением %Считает ведь, да?
		, потому что именно он дал возможность пройти интересный участок с пещерами. Они стали красивой финальной точкой в походе;
		\item Судя по комментариям участников, раскладку стоило сделать сытнее;
		\item Поскольку в группе было всего четыре участника, удельный вес общественного снаряжения на участника, был велик. Как следствие, вырос вес рюкзаков, в сравнении с ожидаемым. В рамках этого маршрута наличие веревки и систем %По крайней мере, мне так показалось. Брод проходится без веревки, остальное - тоже.
		в общественном снаряжении является спорным вопросом;
		\item При подготовку к маршрутам, одной из целей которых является «посмотреть красивое» стоит узнавать интересные факты об этом самом «красивом». Иными словами, в этом походе хотелось бы видеть человека с должностью Краевед, который бы разведал, а потом рассказал, откуда взялись пещеры горы Улукая, почему Калинов мост проходит не через реку Смородину и что из себя представляет Немецкий аэродром с точки зрения истории;
		\item Решение, принятое касательно времени движения, считаю хорошим и правильным. Благодаря подьёмам в 4:00 - 4:30 получалось захватить наибольшую часть хорошей погоды и был куда более широкий простор для выбора времени старта подходящего места ночёвки. Этот момент был плохо раскрыт в прошлых походах. % Тоже не уверена, нужно ли такое в выводах писать
		
		%% Комментарии про состав раскладки?
	\end{enumerate} 

	\clearpage