\subsection{10 августа. Пер. Иттиш (1A)}
\textit{Метеоусловия: утром, днём, вечером ясно, тепло. В середине дня жарко.}

\begin{figure}[h!]
	\centering
		\includegraphics[angle=0, width=0.4\linewidth]{pics/maps/10.png}
	\label{fig:mini_10}
\end{figure}

Подъем дежурных в 05:00, выходим с м.н. 6 в 07:15. К 10:30 преодолеваем подъем по морене и выходим на горизонтальную осыпь на высоте $\sim$3700м. Оставшуюся часть дня двигаемся по этой осыпи.

\begin{figure}[h!]
	\centering
	\includegraphics[width=0.7\linewidth]{pics/10/IMG_3429}
	\caption{Осыпь на подходе к пер. Иттиш}
	\label{fig:IMG_3429}
\end{figure}

Горизонтальная осыпь состоит из средних и крупных камней. В некоторых местах двигаемся по травянистым участкам, в некоторых~--- по пересохшей части дна озер, в некоторых приходится двигаться лазанием по крупным камням. Для ускорения рекомендуем ориентироваться на установленные туры, избегать лазания, пользоваться движением по дну озер, где это возможно. Однозначного рецепта преодоления данной осыпи нет.

\begin{figure}[h!]
	\centering
	\includegraphics[width=0.7\linewidth]{pics/10/IMG_3421}
	\caption{Движение по пересохшему дну озера}
	\label{fig:IMG_3421}
\end{figure}

В самое жаркое время суток, с 13:00 до 15:00, устраиваем обед вблизи ручья, впадающего в одно из озер (N 42.02672\degree~ E 77.98596\degree). После этого идём часть пути по дну озера и далее двигаемся по гребням морен. Морены состоят из камней среднего размера.

\begin{figure}[h!]
	\centering
	\includegraphics[width=0.7\linewidth]{pics/10/IMG_3489}
	\caption{Группа на седловине пер. Иттиш. Вид на сырты и хр. Акшийрак.}
	\label{fig:IMG_3489}
\end{figure}

На седловину перевала выходим в 17:00, преодолев кулуар, разделяющий моренный гребень с седловиной перевала. При движении по орогр. правому борту, по моренному карману, этот участок камнеопасен, так борт представлен отвесной стеной высотой несколько десятков метров~\cite{kovinov2021}. Однако при движении по гребню опасности поймать камень нет~\cite{tipsina2024} Спуска с перевала нет, сразу идет выход на высокогорные луга -- сырты. Пройдя по горизонтали около 1~км, встаём на ночевку на озере в 18:00, соединившись с группой Кати Тюриной. Координаты м.н.: N 42.00321\degree~ E 77.99574\degree.

\begin{figure}[h!]
	\centering
		\includegraphics[width=0.7\linewidth]{pics/10/camp_10}
	\caption{Место ночёвки 10-11.08}
	\label{fig:camp_10}
\end{figure}

\textbf{Выводы и рекомендации:} пер. Иттиш полностю соответствует заявленной к.тр. 1А. Определяющая сторона~--- северная. Перевал неклассический: перевальными взлётами не обладает, вместо этого с северной стороны~--- длительное движение по долине и широкому пологому кулуару с цепями озёр, а с южной~--- плоскогорье и сырты. Из-за последнего обстоятельства перевал необходимо включать в нитку маршрута не ранее середины похода, чтобы группа была акклиматизирована. Сам перевал необычен и красив из-за принципиальной возможности пересечь хребет Тескей-Ала-Тоо в горном походе 1 к.с., из-за открывающихся видов на сырты и высокие скальные гребни пересекаемого основного хребта. Горячо рекомендуется к посещению в новичковых горных походах.

\clearpage