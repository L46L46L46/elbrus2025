\subsection{07 августа. пер. Перемётный (1А)}
\textit{Метеоусловия: утром, днём ясно, с 15:00 по 19:00 пасмурно, дождь, ночью сильный дождь}

\begin{figure}[h!]
	\centering
	\includegraphics[angle=90, width=0.7\linewidth]{pics/maps/7}
	\label{fig:7}
\end{figure}

Подъём дежурных в 04:30, выход в 07:10. Движемся к левому берегу одного из притоков в сторону небольшого голубого озерца в течение 15 мин ЧХВ и подходим к подножию морен. Приток можно перейти по камням, некоторые участники переходят в бродовой обуви.

\begin{figure}[h!]
	\centering
	\includegraphics[width=0.7\linewidth]{pics/07/IMG_2778}
	\caption{Место брода и дальшейшее движение по морене}
	\label{fig:img2778}
\end{figure}

Далее забираемся на моренный вал, огибаем локальную возвышенность справа пхд (рис.~\ref{fig:img2778}) и следуем по локальному понижению в направлении цирка Лавинных перевалов в течение 30 мин ЧХВ. Уклон не более 15\degree, подниматься комфортно (рис.~\ref{fig:DJI_0059.jpg}).

\begin{figure}[h!]
	\centering
	\includegraphics[width=0.9\linewidth]{pics/07/DJI_0059.jpg}
	\caption{Маршрут подъёма на пер. Перемётный}
	\label{fig:DJI_0059.jpg}
\end{figure}

Выходим на левый пхд борт ледника с осыпного гребня, разделяющего цирки Лавинных перевалов и пер. Перемётный, поскольку подъём с языка ледника без кошек невозможен. Ледник открытый, пологий и проходится без кошек. Его уклон не превышает 10\degree, лишь в верхней его части есть участок крутизной до 15\degree и с поперечными трещинами, которые легко обходятся.

\begin{figure}[h!]
	\centering
	\includegraphics[width=0.9\linewidth]{pics/07/IMG_2857.jpg}
	\caption{Движение по леднику}
	\label{fig:IMG_2857.jpg}
\end{figure}

\begin{figure}[h!]
	\centering
	\includegraphics[width=0.9\linewidth]{pics/07/IMG_2904.jpg}
	\caption{Движение по леднику}
	\label{fig:IMG_2904.jpg}
\end{figure}

\begin{figure}[h!]
	\centering
	\includegraphics[width=0.9\linewidth]{pics/07/view_kiche.jpg}
	\caption{Маршрут подъёма, вид с седловины}
	\label{fig:view_kiche.jpg}
\end{figure}

\begin{figure}[h!]
	\centering
	\includegraphics[width=0.9\linewidth]{pics/07/DJI_0067.jpg}
	\caption{Группа на пер. Перемётный (1А). Вид в д.р. Джукучак}
	\label{fig:DJI_0067.jpg}
\end{figure}

\begin{figure}[h!]
	\centering
	\includegraphics[width=0.9\linewidth]{pics/07/DJI_0069.jpg}
	\caption{Группа на пер. Перемётный (1А). Вид в д.р. Киче-Кызыл-Суу}
	\label{fig:DJI_0069.jpg}
\end{figure}

Координаты сложенного нами тура: N 42.09765° E 78.12405°



\begin{figure}[h!]
	\centering
	\includegraphics[width=0.7\linewidth]{pics/07/IMG_3087.jpg}
	\caption{Спуск по крутому тр. склону в <<коридоре>> между обрывом и скальными  выходами}
	\label{fig:IMG_3087.jpg}
\end{figure}


В 16:50 спускаемся, наконец, на дно д.р. Джукучак и движемся вниз по долине в направлении ночёвок под пер. Ашутор Зап. Спустя 30 минут движения нас накрывает дождём. Решаем вставать на первых же удобных площадках и бродить р. Джукучак утром следующего дня. В итоге выбираем место за 600 м до брода через р. Джукучак. Первые участники подходят под м.н. в 18:30, ставят палатки и греют чай. Остальные подтягиваются к 19:00, поим всех чаем и раскладываем отдыхать. Дежурные также разносят еду по палаткам. Дождь к вечеру прекращается, однако ночью льёт пуще прежнего.
Координаты м.н.: N 42.11030° E 78.05595°. ЧХВ 10:14.

\clearpage