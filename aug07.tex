\subsection{07 августа. пер. Перемётный (1А)}
\textit{Метеоусловия: утром, днём ясно, с 15:00 по 19:00 пасмурно, дождь, ночью сильный дождь}

\begin{figure}[h!]
	\centering
	\includegraphics[angle=90, width=0.7\linewidth]{pics/maps/7}
	\label{fig:7}
\end{figure}

Подъём дежурных в 04:30, выход в 07:10. Движемся к левому берегу одного из притоков в сторону небольшого голубого озерца в течение 15 мин ЧХВ и подходим к подножию морен. Приток можно перейти по камням, некоторые участники переходят в бродовой обуви.

\begin{figure}[h!]
	\centering
	\includegraphics[width=0.7\linewidth]{pics/07/IMG_2778}
	\caption{Место брода и дальшейшее движение по морене}
	\label{fig:img2778}
\end{figure}

Далее забираемся на моренный вал, огибаем локальную возвышенность справа пхд (рис.~\ref{fig:img2778}) и следуем по локальному понижению в направлении цирка Лавинных перевалов в течение 30 мин ЧХВ. Уклон не более 15\degree, подниматься комфортно (рис.~\ref{fig:DJI_0059.jpg}).

\begin{figure}[h!]
	\centering
	\includegraphics[width=0.9\linewidth]{pics/07/DJI_0059.jpg}
	\caption{Маршрут подъёма на пер. Перемётный}
	\label{fig:DJI_0059.jpg}
\end{figure}

Выходим на левый пхд борт ледника с осыпного гребня, разделяющего цирки Лавинных перевалов и пер. Перемётный, поскольку подъём с языка ледника без кошек невозможен. Сам же ледник открытый, пологий и проходится без кошек. Его уклон не превышает 10\degree, лишь в верхней части есть участок крутизной до 15\degree и с поперечными трещинами, которые легко обходятся.

\begin{figure}[h!]
	\centering
	\includegraphics[width=0.7\linewidth]{pics/07/IMG_2857.jpg}
	\caption{Движение по леднику}
	\label{fig:IMG_2857.jpg}
\end{figure}

\begin{figure}[h!]
	\centering
	\includegraphics[width=0.7\linewidth]{pics/07/IMG_2904.jpg}
	\caption{Движение по леднику}
	\label{fig:IMG_2904.jpg}
\end{figure}

\begin{figure}[h!]
	\centering
	\includegraphics[width=0.9\linewidth]{pics/07/view_kiche.jpg}
	\caption{Маршрут подъёма, вид с седловины}
	\label{fig:view_kiche.jpg}
\end{figure}

Движемся по леднику в течение 70 мин ЧХВ и в 11:30 выходим на перевал. Седловина~--- широкое ледовое плато, тура, как и писали в предыдущих отчётах, нет. Складываем свой тур на осыпном гребешке правого пхд борта перевала в надежде, что если группы будут ходить здесь каждый год, то он не будет успевать уезжать вместе с перемётным ледником. Координаты сложенного нами тура: N 42.09765° E 78.12405°.

\begin{figure}[h!]
	\centering
	\includegraphics[width=0.9\linewidth]{pics/07/DJI_0067.jpg}
	\caption{Группа на пер. Перемётный (1А). Вид в д.р. Джукучак}
	\label{fig:DJI_0067.jpg}
\end{figure}

\begin{figure}[h!]
	\centering
	\includegraphics[width=0.9\linewidth]{pics/07/DJI_0069.jpg}
	\caption{Группа на пер. Перемётный (1А). Вид в д.р. Киче-Кызыл-Суу}
	\label{fig:DJI_0069.jpg}
\end{figure}


\begin{figure}[h!]
	\centering
	\includegraphics[width=0.7\linewidth]{pics/07/typ.jpg}
	\caption{Местоположение перевального тура}
	\label{fig:typ.jpg}
\end{figure}

Спускаемся по пологому правому борту ледника до озера в течение 15 минут. Этот участок также проходим без кошек, но трое участников их всё же надевают, скорее, потому что кошки были в наличии. В крайнем случае при спуске есть возможность перейти на правый пхд осыпной борт и спуститься там.

\begin{figure}[h!]
	\centering
	\includegraphics[width=0.7\linewidth]{pics/07/IMG_2970.jpg}
	\caption{Путь спуска с пер. Перемётный}
	\label{fig:IMG_2970.jpg}
\end{figure}

За озером находятся галечные площадки (координаты: N 42.09315° E 78.11896°), на которых, при необходимости, можно встать на ночёвку. Вода из правого притока р. Джукучак, этот ручей как раз берет своё начало здесь. На этом месте мы обедаем и в 13:10 выдвигаемся на спуск. Ручей довольно быстро уходит в систему каньонов (насчитали три штуки), поэтому спускаемся, траверсируя склон правого борта притока. Под ногами мелкая осыпь, но она достаточно крупна, чтобы <<лифтом>> спускаться было не очень удобно. Двигаемся медленно и аккуратно, так как склон крутой и заканчивается каньоном. Кроме того, нам встречаются две селевые промоины, борта которых сцементированы прошедшим потоком. Первая промоина неглубока и перебраться через неё не составило труда, а для преодоления второй пришлось спуститься на 20~м, а потом подняться обратно.

\begin{figure}[h!]
	\centering
	\includegraphics[width=0.7\linewidth]{pics/07/IMG_3031.jpg}
	\caption{Траверс осыпного склона}
	\label{fig:IMG_3031.jpg}
\end{figure}

\begin{figure}[h!]
	\centering
	\includegraphics[width=0.9\linewidth]{pics/07/IMG_3041.jpg}
	\caption{Траверс осыпного склона (вид в сторону перевала)}
	\label{fig:IMG_3041.jpg}
\end{figure}

Далее идёт спуск по крутому (до 30\degree) травянистому склону. Здесь важно не уходить слишком вправо, хоть там склон и положе, так как этот пологий склон оканчивается скальными выходами. Сильно влево забирать также нельзя, так как река снова образует каньон.

\begin{figure}[h!]
	\centering	\includegraphics[width=0.7\linewidth]{pics/07/IMG_3087.jpg}
	\caption{Спуск по крутому тр. склону в <<коридоре>> между обрывом и скальными  выходами}
	\label{fig:IMG_3087.jpg}
\end{figure}


В 16:50 спускаемся, наконец, на дно д.р. Джукучак и движемся вниз по долине в направлении ночёвок под пер. Ашутор Зап. Спустя 30 минут движения нас накрывает дождём. Решаем вставать на первых же удобных площадках и бродить р. Джукучак утром следующего дня. В итоге выбираем место за 600 м до брода через р. Джукучак. Первые участники подходят под м.н. в 18:30, ставят палатки и греют чай. Остальные подтягиваются к 19:00, поим всех чаем и раскладываем отдыхать. Дежурные также разносят еду по палаткам. Дождь к вечеру прекращается, однако ночью льёт пуще прежнего.
Координаты м.н.: N 42.11030° E 78.05595°. ЧХВ 10:14.

\begin{table}[h!]
	\centering
	\begin{tabular}{|c|c|c|c|c|c|} 
		\hline 
		Этап & ЧХВ \\ 	
		\hline 
		Подъём в верховья р. Киче-Кызыл-Суу  & 02:00 \\
		Подход по моренам к леднику  & 03:00 \\
		По леднику до седловины & 01:10\\ 
		По леднику до галечных площадок & 00:50\\ 
		Траверс осыпного склона & 02:40\\ 
		Спуск по травянистому склону в д.р. Джукучак & 00:35 \\
		\hline
		\textsc{Полное время подъёма на перевал  }& 06:10\\
		\textsc{Полное время спуска с перевала }& 04:05 \\
		\textsc{	Полное время прохождения перевала }& 10:15 \\
		\hline
	\end{tabular}
	\caption{Расклад времени, пер. Перемётный}
\end{table}



\textbf{Выводы и рекомендации:} пер. Перемётный соответствует заявленной категории трудности 1А (при условии открытого ледника Киче-Кызылсу), достаточно красив и интересен. Подъём на перевал не представляет сложности, интересен распутыванием несложного моренного лабиринта и длинным пологим ледником. Последнее обстоятельство особенно ценно: не в каждом походе 1 к.с. удастся вдоволь погулять по леднику. Спуск по долине правого притока р. Джукучак долог и утомителен: сначала 2 км траверса малко-средней сыпухи, ведущей в ущелье каньона, затем~--- лавирование по достаточно крутому травянистому склону между скальными сбросами и ещё более крутым склоном. Для новичков такие препятствия могут быть выматывающими при прохождении, это нужно учитывать при планировании ходового дня. Перевал не рекомендуется ставить первой <<единичкой А>> на маршруте и проходить его в обратную сторону (с юга на север). 

\clearpage