\subsection{07 августа. пер. Перемётный (1А)}
\textit{Метеоусловия: утром, днём ясно, с 15:00 по 19:00 пасмурно, дождь, ночью сильный дождь}

\begin{figure}[h!]
	\centering
	\includegraphics[angle=90, width=0.7\linewidth]{//webdav.cloud.mail.ru@SSL/DavWWWRoot/maps/7}
	\label{fig:mini_18}
\end{figure}

Подъём дежурных в 04:30, выход в 07:10. Движемся к левому берегу одного из притоков в сторону небольшого голубого озерца в течение 15 мин ЧХВ и подходим к подножию морен. Приток можно перейти по камням, некоторые участники переходят в бродовой обуви.

\begin{figure}[h!]
	\centering
	\includegraphics[width=0.7\linewidth]{//webdav.cloud.mail.ru@SSL/DavWWWRoot/07/IMG_2778}
	\caption{Место брода и дальшейшее движение по морене}
	\label{fig:img2778}
\end{figure}

Далее забираемся на моренный вал, огибаем локальную возвышенность справа пхд (рис.~\ref{fig:img2778}) и следуем по локальному понижению в направлении цирка Лавинных перевалов в течение 30 мин ЧХВ.
\begin{figure}[h!]
	\centering
	\includegraphics[width=0.9\linewidth]{//webdav.cloud.mail.ru@SSL/DavWWWRoot/07/DJI_0059.jpg}
	\caption{Маршрут подъёма на пер. Перемётный}
	\label{fig:DJI_0059.jpg}
\end{figure}

\begin{figure}[h!]
	\centering
	\includegraphics[width=0.9\linewidth]{//webdav.cloud.mail.ru@SSL/DavWWWRoot/07/DJI_0067.jpg}
	\caption{Группа на пер. Перемётный (1А). Вид в д.р. Джукучак}
	\label{fig:DJI_0067.jpg}
\end{figure}

Координаты сложенного нами тура: N 42.09765° E 78.12405°

\clearpage