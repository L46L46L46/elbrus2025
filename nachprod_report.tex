\section{Финансовый отчёт}

Раскладка была составлена на группу из 8-ми человек и 13 дней, включая два запасных. Средний вес 
продуктов в сухом виде на одного человека составил 536 гр./ 1943 ккал. в день, суммарный вес продовольствия на человека~--- 6.3 кг (без учёта заброски).
Заброска была спланирована на старте, выброска — вечером пятого дня. Участникам было рекомендовано брать 60 или более г карманки на день.
\\\\
Раскладка была составлена следующим образом:\\
...\\
Сделанные выводы:
\begin{itemize}
    \item Раскладку можно было делать "голоднее", еды было много даже с учётом излишка некоторых видов продуктов из-за слива трёх участников.
    \item Омлета нужно было закладывать 40-50 граммов вместо 30-ти, тогда это оправдало бы его постановку на завтрак наиболее тяжёлых дней.
    \item На ужин, по просьбам медика, пили вопреки раскладке иванчай вместо чая зелёного/чёрного как наименее кофеинсодержащий напиток. В будущем 
имеет смысл сразу ставить на ужин различные успокаивающие травяные чаи.
    \item Какао и кисель пили не все.
    \item С огромным удовольствием участники ели карпюр с пеммиканом и жареным луком: в одном кане на пеммикановом жиру жарится лук, в другом — 
кипяток для разведения карпюра и чая. Изначальная нелюбовь начпрода к карпюру из-за его малой калорийности на единицу веса не выдержала испытания 
практикой.
    \item Кисель и какао идут хуже чаёв, ещё и каны после них отмывать.
    \item Киноа долго варится.
    \item Банан не очень хорош в качестве досыпки в кашу, т.к. пресный, без кислинки или тому подобной изюминки.
    \item Курут~---  на любителя.
    \item В несколько особо тяжёлых дней на суповарение не было времени, и 
довольствовались сухой частью обеда. Поэтому имеет смысл закладывать сухой обед как минимум на запасные дни, чтобы в таких случаях заменять "мокрый"
обед полноценным сухим запасного дня.
    
\end{itemize}


%—————————————————————————————————————————————————
% С чего скатываю:
%Раскладка была составлено следующим образом: 
%− на завтрак были заложены гречка, булгур, рис, макароны или пюре (по 250г на 
%четверых) и 250 грамм тушеной свинины или говядины 
%− на обед брикетированный суп лидкон (гороховый, лидский, рассольник или харчо) 
%1 пачка 200г на четверых, 200г колбасы или паштета, 100г хлебцов 
%− на ужин заложены гречка, булгур, рис, макароны или пюре (по 250г на четверых), 
%325г тушеной говядины, тушеной свинины и 50г сала 
%− каждый прием пищи предполагал 4л чая 
%− на день выделялось 150г козинаков и 45г лимона 
%− каждому участнику на поход было выделено по 6 энергетических гелей GU Original 
%По результатам питания в походе можно сделать следующие выводы: 
%− главная проблема раскладки – однообразие. С учетом плохого аппетита во время 
%акклиматизации меню должно быть разнообразным и вкусным; 
%− супы лидкон не годятся к использованию в качестве единственного блюда на обед 
%каждый день; на вкус супы однообразны и быстро надоедают; к концу похода супы 
%не варили, на обед ограничивались колбасой/паштетом с хлебцами; 
%− тушенка «Кранидов» качественная и вкусная, но употребление ее два раза в день к 
%середине похода надоедает, поэтому включать в дневную раскладку ее стоит не 
%более раза; 
%− сладкого имеет смысл брать больше, козинаки съедали с удовольствием; 
%− самым вкусным блюдом, которое если все участники с удовольствием было пюре с 
%тушенкой 
%− энергетические гели произвели двоякое впечатление на участников – кому-то они 
%помогали восстанавливать силы, на кого-то не оказывали влияния совсем; 
%необходимо каждому участнику группы индивидуально до похода проверять 
%работоспособность таких гелей. 
% 
%В целом приходится отметить, что раскладка была недостаточно проработана. Главная 
%проблема – однообразие. Вероятно, такая раскладка была бы удовлетворительной без учета 
%влияния акклиматизации на аппетит, но для использования на высокогорных маршрутах ее 
%обязательно нужно переработать.
