\section{Отчёт завхоза}

Продуктовая раскладка была позаимствована из августовского похода по Терскею \cite{teskei2025ostapiv} и несколько урезана. Именно, сыр на завтрак и хлебцы на обед и ужин были не каждый день, а через день. Граммовки сухмяса, пеммикана, суховощей были снижены на 20\%. Сделано это было с целью облегчения веса рюкзаков и поиска оптимума раскладки, т.к. в предыдущих походах, по мнению участников, раскладку можно было сделать <<тоще>>. Кроме того, имелось в виду, что физические нагрузки будут ниже, чем в обычной горной <<единичке>>

Что получилось в итоге:
\begin{itemize}
    \item По мнению одного участника из четырёх, раскладка была слишком <<тощей>>, хотя, по его же мнению, это можно было компенсировать более калорийной карманкой. Остальнм троим было приемлемо.
    \item С огромным удовольствием участники ели карпюр с пеммиканом и жареным луком: в одном кане на пеммикановом жиру жарится лук, в другом — 
	кипяток для разведения карпюра и чая.
	\item Чая много не бывает. На полуднёвках, под затяжной дождь, чай заходил очень хорошо.
    
\end{itemize}


%—————————————————————————————————————————————————
% С чего скатываю:
%Раскладка была составлено следующим образом: 
%− на завтрак были заложены гречка, булгур, рис, макароны или пюре (по 250г на 
%четверых) и 250 грамм тушеной свинины или говядины 
%− на обед брикетированный суп лидкон (гороховый, лидский, рассольник или харчо) 
%1 пачка 200г на четверых, 200г колбасы или паштета, 100г хлебцов 
%− на ужин заложены гречка, булгур, рис, макароны или пюре (по 250г на четверых), 
%325г тушеной говядины, тушеной свинины и 50г сала 
%− каждый прием пищи предполагал 4л чая 
%− на день выделялось 150г козинаков и 45г лимона 
%− каждому участнику на поход было выделено по 6 энергетических гелей GU Original 
%По результатам питания в походе можно сделать следующие выводы: 
%− главная проблема раскладки – однообразие. С учетом плохого аппетита во время 
%акклиматизации меню должно быть разнообразным и вкусным; 
%− супы лидкон не годятся к использованию в качестве единственного блюда на обед 
%каждый день; на вкус супы однообразны и быстро надоедают; к концу похода супы 
%не варили, на обед ограничивались колбасой/паштетом с хлебцами; 
%− тушенка «Кранидов» качественная и вкусная, но употребление ее два раза в день к 
%середине похода надоедает, поэтому включать в дневную раскладку ее стоит не 
%более раза; 
%− сладкого имеет смысл брать больше, козинаки съедали с удовольствием; 
%− самым вкусным блюдом, которое если все участники с удовольствием было пюре с 
%тушенкой 
%− энергетические гели произвели двоякое впечатление на участников – кому-то они 
%помогали восстанавливать силы, на кого-то не оказывали влияния совсем; 
%необходимо каждому участнику группы индивидуально до похода проверять 
%работоспособность таких гелей. 
% 
%В целом приходится отметить, что раскладка была недостаточно проработана. Главная 
%проблема – однообразие. Вероятно, такая раскладка была бы удовлетворительной без учета 
%влияния акклиматизации на аппетит, но для использования на высокогорных маршрутах ее 
%обязательно нужно переработать.
