\subsection{09 сентября. пер. Бересун (н/к)}
\textit{Метеоусловия: утром морось, днём, вечером дождь}

\begin{figure}[h!]
	\centering
	\includegraphics[angle=90, width=0.8\linewidth]{pics/maps/09}
	\label{fig:09}
\end{figure}

Встаём в 04:45, завтракаем. Идёт дождь, поэтому ждём, пока он ослабнет. Выходим в 07:35. Накануне прочли в спортмарафоновский статье \cite{dzhilysu}, что в Джилы-Су можно пройти как по грунтовке, так и по тропе с переходом р. Кызылкол по природному Калинову мосту. Решаем воспользоваться этой возможностью. Дорога от альплагеря до моста занимает 40~мин ЧХВ. Виды вокруг в сочетании с погодой сногсшибательные: мелкий дождик, солнышко, радуга, всё блестит и переливается, а водопады и этот природный мост переносят в атмосферу сказки. Определённо, это место оказалось одним из красивейших за поход.

\begin{figure}[h!]
	\centering
	\includegoogle[width=0.8\linewidth]{https://drive.google.com/file/d/1653Md_fZRgcozecS7IMmwZSWjrypf16l/view?usp=drive_link}
	\caption{Калинов мост}
	\label{fig:kalinov}
\end{figure}

Далее спускаемся в д.р. Малка к нарзанам и подножию водопада Султан, переход занимает не более 15 мин ЧХВ. Здесь уже много народу, который приезжает на нарзаны и термальные источники одним днём. Мы же устраиваем фотосессию на фоне водопада и движемся дальше. 

\begin{figure}[h!]
	\centering
	\includegoogle[width=0.6\linewidth]{https://drive.google.com/file/d/16o3NzIGHWwUXX38eic0tMJFodG6oBt-p/view?usp=drive_link}
	\caption{Фоткаемся на водопаде Султан}
	\label{fig:sultan}
\end{figure}

\begin{figure}[h!]
	\centering
	\includegoogle[width=0.5\linewidth]{https://drive.google.com/file/d/1r2O1579I7LmHdWf9USunYBycIDd5BUzm/view?usp=drive_link}
	\caption{Сам водопад Султан}
	\label{fig:sultan1}
\end{figure}

После перехода р.~Малка по мосту N 43.43353° E 42.53515° с удивлением обнаруживаем табличку <<Запрещено! Опасно для жизни!>>, обращённую лицевой стороной в цивилизацию, то есть против направления нашего движения. Наконец, в 09:30 доходим до кафешки N 43.43495° E 42.53717°, где балуемся хычинами. Хычин с картошкой и сыром (а других не было) стоит 200~\faRub.~В 09:50 выдвигаемся по асфальтовой дороге к кемпингу <<Горный край>>, где нас ждёт заброска.

На базу (N 43.44527° E 42.55451°) приходим в 10:20. За хранение заброски (3 дня) с нас взяли 1000~\faRub.~Вкупе с тем, что стоимость ее доставки была 10000~\faRub, а суммарный ее вес не превышал 7 кг, вопрос о целесообразности этой заброски является крайне дискуссионным. Как бы то ни было, за 20 минут раскладываем еду по рюкзакам и выдвигаемся на перевал. Почти сразу за базой асфальтовая дорога Джилы-Су уходит на другой борт долины, а мы поворачиваем направо пхд, на неплохую грунтовку, и начинаем подъём на перевал. Накрапывает дождь, опустился густой туман, обстановка атмосферная.


\begin{figure}[h!]
	\centering
	\includegoogle[width=0.7\linewidth]{https://drive.google.com/file/d/1H5EXn4Z_yx4gIJJIm6J66QT36v3Q5Fns/view?usp=drive_link}
	\caption{Путь подъёма на пер. Бересун}
	\label{fig:beresun1}
\end{figure}


В 11:35 поднимаемся на пер. Бересун. Фактически, это даже не перевал, а траверс отрога хребта, поcкольку у него нет седловины. Льёт дождь, всё заволокло туманом. Тура нет, но мы полны решимости исправить эту несправедливость и складываем его. Координаты тура: N 43.46051° E 42.57026°. Делаем переполненные технической информацией фотографии на обе стороны перевала (рис.~\ref{fig:beresun2}, \ref{fig:beresun3}) и в 11:45 начинаем спуск в д.р. Шаукам.

\begin{figure}[h!]
	\centering
\begin{minipage}[t]{0.49\textwidth}
	\centering
	\includegoogle[width=0.99\linewidth]{https://drive.google.com/file/d/1ND1RBTf4EPR06560QBWaumla-9EzW3hS/view?usp=drive_link}
	\subcaption{Группа на пер. Бересун, <<вид>> на д.р. Шаукам}
	\label{fig:beresun2}

\end{minipage}
\hfill
\begin{minipage}[t]{0.49\textwidth}
		\centering
		\includegoogle[width=0.99\linewidth]{https://drive.google.com/file/d/1vLVC9UjNFny8Y7SFxwtWrAkokeYooxZO/view?usp=drive_link}
		\subcaption{Группа на пер. Бересун, <<вид>> на д.р. Малка}
		\label{fig:beresun3}
	
\end{minipage}
\end{figure}
Спуск также проходит по автодороге, однако с обеих её сторон растут деревья, что, определённо, добавляет уюта. 

\begin{figure}[h!]
	\centering
	\includegoogle[width=0.7\linewidth]{https://drive.google.com/file/d/12HzfnQx7EL9430Vsvcg2n0kPx0oREA_1/view?usp=drive_link}
	\caption{Спуск с пер. Бересун, <<вид>> на д.р. Шаукам}
	\label{fig:shaukam}
\end{figure}

В 12:30 спускаемся на дно долины, дорога идёт вверх по течению. Начинается сильный дождь, поэтому мы ищем подходящее м.н. Ближайшее находится у моста через р. Шаукам незадолго до впадения в неё р. Исламчат, у коша. Встаём здесь в 12:45. Здесь грунтовка забирает налево пхд, в д.р. Шаукам, далее по Большой Кавказской тропе, а нам необходимо будет двигаться вверх по д.р. Исламчат.

Вода в р. Шаукам после многодневных дождей мутная, но, к счастью, в 100~м от лагеря на склоне левого берега р. Исламчат есть выход небольшого источника (рис.~\ref{fig:mn910}). Судя по следам козьих копыт, источник интересен не только нам.

\begin{figure}[h!]
	\centering
	\includegoogle[width=0.9\linewidth]{https://drive.google.com/file/d/1PlUo5aYbtegnL4ziz-yclVW-Yc4lV9C7/view?usp=drive_link}
	\caption{Место ночёвки с указанием близлежащих дорог}
	\label{fig:mn910}
\end{figure}

Коориднаты м.н.: N 43.43946° E 42.60685°. ЧХВ: 03:34 , ОХВ: 5:10
\clearpage
