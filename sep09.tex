\subsection{09 сентября. пер. Бересун (н/к)}
\textit{Метеоусловия: морось}

\begin{figure}[h!]
	\centering
	\includegraphics[angle=90, width=0.8\linewidth]{pics/maps/09}
	\label{fig:09}
\end{figure}

Подъём дежурных в 04:45, выход в 07:35. Движемся по хорошей тропе 42 мин без привала и подходим к водопаду Султран. Любуемся видами и фотографируемся до 8:32. в 9:28 выходим в цивилизацию рядом с нарзанными источниками и общественнными нарзанными ваннами по мосту, на обращённом к цивилизации выходе с которого стоит табличка "Запрещено! Опасно для жизни" (мы с другого конца не могли её увидеть). Делаем привал на поедание хычынов. Ещё через 1:40 движения по уходящей в туман двухколейной дороге достигаем пер. Бересун (2479) и 20-минут отдыхаем. Красотами полюбоваться не удалось из-за тумана и завесы мороси. В 11:45 начинаем спуск и в 13:00 встаём на место ночёвки, координаты (???). ЧХВ составило 4:33.


\begin{figure}[h!]
	\centering
	\includegoogle[width=0.7\linewidth]{https://drive.google.com/file/d/1H5EXn4Z_yx4gIJJIm6J66QT36v3Q5Fns/view?usp=drive_link}
	\caption{Путь подъёма на пер. Бересун}
	\label{fig:beresun1}
\end{figure}


\begin{figure}[h!]
	\centering
	\includegoogle[width=0.7\linewidth]{https://drive.google.com/file/d/1ND1RBTf4EPR06560QBWaumla-9EzW3hS/view?usp=drive_link}
	\caption{Группа на пер. Бересун, <<вид>> на д.р. Шаукам}
	\label{fig:beresun2}
\end{figure}

\begin{figure}[h!]
	\centering
	\includegoogle[width=0.7\linewidth]{https://drive.google.com/file/d/1vLVC9UjNFny8Y7SFxwtWrAkokeYooxZO/view?usp=drive_link}
	\caption{Группа на пер. Бересун, <<вид>> на д.р. Малка}
	\label{fig:beresun3}
\end{figure}



\textbf{Выводы и рекомендации:} пер. Бересун соответствует заявленной категории трудности н/к.  Подъём на перевал не представляет сложности. 

\clearpage
